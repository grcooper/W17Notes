\documentclass[12pt]{article}

\setlength\parindent{0pt}
\newcommand{\myt}[1]{\textbf{\underline{#1}}}

\usepackage{mathtools}
\usepackage{amssymb}
\usepackage{graphicx}

\title{\vspace{-15ex}Introduction to Dimensional Analysis\vspace{-1ex}}
\date{January 20th, 2017}
\author{Graham Cooper}

\begin{document}
	\maketitle
	
	In applications we'll want to keep track of the units of measurement. For cimplicity we'll speak instead of the "dimensions" being measured.\\
	New notation: The dimensions of mass are mass - square brackets define the dimensions of something\\
	$[mass] = M$, $[length] = L$, $[time] = T$ etc. We will add more later\\
	
	We start with two axioms:\\
	\begin{enumerate}
		\item (D1) Physical quantities may only be added ,subtracted, or equated if they have the same dimensions.
		\item (D2) Quantities of different dimensions can only be combined by multiplication and division, in which case we have $[AB] = [A][B]$ and $[\frac{A}{B}] = \frac{[A]}{[B]}$
	\end{enumerate}

	We can define dimensions for any quatntites which we believe shoudl obey these rules\\
	In Physichs applications, we have 5 dimensions:\\
	M, L, T and $[temperature] = U$ and $[charge] = Q$\\
	(We can also define our own dimensions $[money]$, or $[applies]$ and $[oranges]$ etc.)\\
	
	We can use D2 to calculate dimensions of secondary quantities:\\
	eg) $[speed] = \frac{[length]}{[time]} = LT^{-1}$\\
	from: $v = \frac{ds}{dt} = \lim_{\Delta t}\frac{\Delta s}{\Delta t}$\\
	
	$[acceleration] = [\frac{dv}{dt}] = \frac{[speed]}{[time]} = LT^{-2}$\\
	$[force] = [mass][acceleration] = MLT^{-2}$\\
	$[work] = [force][distance] = ML^2T^{-2}$\\
	$[voltage] = \frac{[work]}{[charge]} = ML^2T^{-2}Q^{-1}$\\
	
	Comment: Angles are.... \underline{dimensionless}! (in radians, $\Theta = \frac{s}{r})$ with radius r and length of the side is s. so $[\Theta] = \frac{[s]}{[r]} = \frac{L}{L} = 1$\\
	
	\begin{itemize}
		\item There are some theoretical questions about what ahs dimensions and what doesn't (eg angles! - we'll treat angles as dimensionless, since $\Theta = \frac{s}{r})$
		\item A consequence of D1 and D2 is that the input and ouput of any transcendental function must be dimensionless
	\end{itemize}

	Why? Suppose f(x) has a Maclaurin series. Then $f(x) = a_0 + a_1x + a_2x^2 + ... $\\
	
	Eg: $e^x = 1 + x + \frac{x^2}{2} + \frac{x^3}{6} + ...$\\
	
	\subsection*{Application 1: Consistency checks.}
	
	\underline{Example:} In the sky-diver problem, we started with $m\frac{dv}{dt} = -\alpha v-mg$\\
	
	We know $[m] = M$, $[v] = LT^{-1}$, $[t] = T$, $[g] = LT^{-2}$\\
	We can determine $[\alpha]$: the force due to air resistance is $[-\alpha v] = [force]$, so $[\alpha] = \frac{[force]}{[v]} = \frac{MLT^{-2}}{LT^{-1}} = MT^{-1}$\\
	Also $[mg] = MLT^{-2}$\\
	$[m\frac{dv}{dt}] = \frac{MLT^{-1}}{T} = MLT^{-2}$\\
	
	$\frac{dv}{dt} = \lim\limits_{\Delta t \rightarrow 0}\frac{\Delta v}{\Delta t} = \frac{LT^{-1}}{T} = LT^{-2}$\\
	
	Now we found the solution to be:\\
	$v = \frac{mg}{\alpha}(e^{\frac{-\alpha t}{m}} - 1)$\\
	Check? $[v] = LT^{-1}$\\
	$[\frac{-\alpha}{m}t] = \frac{[\alpha][t]}{[m]} = \frac{(MT^{-1})T}{M} = 1$\\
	$[\frac{mg}{\alpha}] = \frac{MLT^{-2}}{MT^{-1}} = LT^{-1}$\\
	
	\subsection*{Application 2: Nondimentionalization of DEs}
	
	We may be able to introduce dimensionless variables, which may allow us to write our DE's in simpler forms.\\
	Eg 1: For a pendulum, we might say that the period is a "characteristic time ( for that pendulum), $t_c$. We could then define a dimensionless time variable, $\tau$ as $\tau = \frac{t}{t_c}$ (after 10 oscillations we'll have $\tau = 10$)\\
	
	\myt{Procedure for Nondimensionalization}\\
	\begin{enumerate}
		\item List the physical constants in the problem, and identify their dimensions.
		\item Make a seperate list for the variables
		\item Find combinations of the constants which have the same dimensions as the varaibles (do this for each variable). These will be the \underline{characteristic scales} and we will then define $\tau = \frac{t}{t_c}$, $\mu = \frac{m}{m_c}$, $\lambda = \frac{l}{l_c}$, $\epsilon = \frac{x}{x_c}$ etc
		\item rewrite the DE and IVP in terms of the new variables (using the chain rule). Note: the char schales will often have simple physical interpretations
	\end{enumerate}
	
	\underline{Example: The Mixing Tank Problem}\\
	Consider a tank holding a mass m(t) of a chemical dissolved in a volume V of water. A solution with concentration C of the same chemical enters the tank at a rate f\\
	
	The contents of the tank are mixed constantly, and the mixed solution exits the tank at $f \frac{L}{min}$. Find m(t)\\
	
	We must have 
	$$\frac{dm}{dt} = (rate in) - (rate out)$$
	$$= fc - f\frac{m}{V}$$
	$$\frac{dm}{dt} = fc - \frac{fm}{v} \_|\_ m(0) = m_0$$
	Consistency check?\\
	$[\frac{dm}{dt}] = MT^{-1}$\\
	$[fc] = [L^3T^{-1}] \times [ML^{-3}] = MT^{-1}$\\
	$[\frac{fm}{v}] = \frac{[L^4T^{-1}][M]}{L^3} = MT^{-1}$\\
	
\end{document}
