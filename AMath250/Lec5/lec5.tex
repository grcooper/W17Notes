\documentclass[12pt]{article}

\setlength\parindent{0pt}
\newcommand{\myt}[1]{\textbf{\underline{#1}}}

\usepackage{mathtools}
\usepackage{amssymb}
\usepackage{graphicx}

\title{\vspace{-15ex}Introduction to Mathematical Modelling\vspace{-1ex}}
\date{January 16th, 2017}
\author{Graham Cooper}

\begin{document}
	\maketitle
	
	\subsection*{The Logistic Model of Population Growth}
	In applications we often have info about the \underline{rate of change} of a quantity\\
	Eg. Consider a population of organisms, P(t), with unlimited resources ( space, food, etc )\\
	That is, $\frac{dP}{dt} = aP$ for some $a \epsilon R$\\
	$$\rightarrow P(t) = Ce^{at}$$
	At $t=0$ we have $P(0) = C$, so C is the initial population.\\
	$$P(t) = P_0e^{at}$$
	(this is the Malthusian model of population growth)\\
	Malthus suggested including a "carrying capacity", K (a maximum sustainable population). How might we modify the equation?\\
	
	One way: \underline{The Logistic Model}\\
	We should alter it in such a way that the dervitive is 0 when we reach K\\
	$$\frac{dP}{dt} = aP(1-\frac{P}{k})$$
	
	\myt{Now Jan 18th:}\\
	Summary from last day:\\
	$$\frac{dP}{dt} = aP(1-\frac{P}{k})$$
	where a is a proportionality constant, and K is the carrying capacity\\
	(Note that $\frac{dP}{dt} \approx aP$ when $P << K$ and $\frac{dP}{dt} \approx 0$ when $P \approx k$)\\
	
	Solution?\\
	$$\int \frac{dP}{P(1-\frac{P}{k})} = \int adt$$
	$$\int \frac{dP}{P(1-\frac{P}{k})} = \int \frac{k}{P(k-P)}dP = ...$$
	we find:\\
	$$P(t) = \frac{k}{Ce^{-at} + 1}$$
	$$P(0) = P_0 \rightarrow P(t) = \frac{kP_0}{(k-P_0)e^{-at} + P_0}$$
	
	\subsection*{DE's Arising from Physical Laws}
	
	\subsubsection*{Newton's 2nd law of motion}
	For relatively smal velocities ($v << c \approx 3\times 10^8$m/s) this states that $\frac{d}{dt}(mv) = F$. m = mass, v = speed, f = net force\\
	
	If m is constant we have $m\frac{dv}{dt} = F$, ie $F = ma$\\
	
	Since $V = \frac{dx}{dt}$ (if x is displacement! we may also write $m\frac{d^2x}{dt^2} = F$\\
	(See fig5.1)\\
	
	\myt{Example: The sky diver problem:}\\
	This is in the course notes, but we are setting it up slightly different.\\
	An object of mass m, in free fall is subject to forces of gravity and air resistance (drag).\\
	I will treat (slightly different than course notes) "up" will be the positive direction so x(t) is height above the ground.\\
	Let $v(t) = \frac{dx}{dt}$\\
	Consider the forces:\\
	Gravity: $F_g = -mg$ ($g \approx 9.8$m/$s^2$)\\
	Air Resistance: Complicated. We will simply assume $F_{air} = -\alpha v$\\
	
	Combining these, $F = F_g + F_{air} = -mg-\alpha v$\\
	$\rightarrow m\frac{dv}{dt} = -mg-\alpha v$\\
	
	ie $\frac{dv}{dt} + \frac{\alpha}{m}v = -g$\\
	We have $V_c = Ce^{\frac{-\alpha}{m}t}$\\
	For $V_p$? Try V = A, we guess a constant (A zero'th order polynomial) then $v' = 0$\\
	so $\frac{\alpha}{m}A = -g$\\
	So $A = \frac{-mg}{\alpha}$\\
	
	Thus, $v(t) = Ce^{\frac{-\alpha}{m}t} - \frac{mg}{\alpha}$\\
	If the object was dropped from rest, then $v(0) = 0$, and so $0 = C - \frac{mg}{\alpha}$, so $C = \frac{mg}{\alpha}$, and so $v(t) = \frac{mg}{\alpha}(e^{\frac{-\alpha}{m}t} - 1)$\\
	(see Fig5.2)
	
	\subsubsection*{Circuit Analysis}
	A simple circuit containing a resistor and a capacitor may be illustrated like this: (fig 5.3). Here, V(t) is a source voltage (could be constant for a battery, eg) and i(t) is the current. R is the resistance (in ohms) of a resistor and C is the capacitance (in Farads) of a capacitor.\\
	
	We have a set of def'us/experimental laws which govern these circuits:\\
	\myt{Kirchhoff's Voltage Law:}\\
	$$v(t) = V_R(t) + V_C(t)$$
	$V_R$ and $V_C$ are the voltage drops (losses of potential entery) at RPC\\
	\myt{Ohm's law}:\\
	$$V_R(t) = iR$$
	\myt{Definition of C}\\
	$$V_c(t) = \frac{1}{C}q$$
	where q(t) is the charge at the capacitor\\
	$$q(t) = \int_{0}^{t}i(t)dt$$
	ie $i(t) = \frac{dq}{dt}$\\
	
	\myt{From Jan 20th}:\\
	Combining these:\\
	$$V = V_R + V_C$$
	$$= R_i(t)$$
	$$\rightarrow V = R\frac{dq}{dt} + \frac{q}{c}$$
	That is:\\
	$$\frac{dq}{dt} + \frac{q}{RC} = \frac{V}{R}$$
	(A first order linear DE for the charge on the capicitor!)\\
	Let's assume V is constant ( the source voltage is a battery).\\
	The solution to the homogeneous:\\
	$$q_h = Ke^{\frac{-2}{RC}}$$
	A particular solution:\\
	We try $q = A$ Then $q' = 0$, so we have $0 + \frac{A}{RC} = \frac{V}{R}$ so $A = VC$\\
	
	$$\rightarrow q(t) = Ke^{\frac{-t}{RC}} + VC$$
	If $q(0) = 0$ then $0 = k + VC$ so $K = -VC$\\
	and so $q(t) = VC[1 - e^{\frac{-t}{RC}}]$\\
	
	See Fig5.4\\
	Therefore $i(t) = \frac{dq}{dt} = \frac{V}{R}e^{\frac{-2}{RC}}$\\
	See Fig5.5\\
	
	Also $V_R(t) = Ve^{\frac{-2}{RC}}$ and $V_C(t) = V[1-e^{\frac{-t}{RC}}]$\\
	See fig5.6\\
	
\end{document}
