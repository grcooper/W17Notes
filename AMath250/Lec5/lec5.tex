\documentclass[12pt]{article}

\setlength\parindent{0pt}
\newcommand{\myt}[1]{\textbf{\underline{#1}}}

\usepackage{mathtools}
\usepackage{amssymb}
\usepackage{graphicx}

\title{\vspace{-15ex}Introduction to Mathematical Modelling\vspace{-1ex}}
\date{January 16th, 2017}
\author{Graham Cooper}

\begin{document}
	\maketitle
	
	In applications we often have info about the \underline{rate of change} of a quantity\\
	Eg. Consider a population of organisms, P(t), with unlimited resources ( space, food, etc )\\
	That is, $\frac{dP}{dt} = aP$ for some $a \epsilon R$\\
	$$\rightarrow P(t) = Ce^{at}$$
	At $t=0$ we have $P(0) = C$, so C is the initial population.\\
	$$P(t) = P_0e^{at}$$
	(this is the Malthusian model of population growth)\\
	Malthus suggested including a "carrying capacity", K (a maximum sustainable population). How might we modify the equation?\\
	
	One way: \underline{The Logistic Model}\\
	We should alter it in such a way that the dervitive is 0 when we reach K\\
	$$\frac{dP}{dt} = aP(1-\frac{P}{k})$$
	
	
\end{document}
