\documentclass[12pt]{article}

\setlength\parindent{0pt}
\newcommand{\myt}[1]{\textbf{\underline{#1}}}

\usepackage{mathtools}
\usepackage{amssymb}
\usepackage{graphicx}

\title{\vspace{-15ex}Lecture 11\vspace{-1ex}}
\date{March 3rd, 2017}
\author{Graham Cooper}

\begin{document}
	\maketitle
	
	Missed a couple of classes...
	
	\section*{Working with Piecewise-Defined Functions}
	To extend our method to IVPs with piecewise-defined functions, we need a simple, \underline{standard} way of expressing them. We will use this:\\
	
	\subsection*{Definition: The Heaviside function (Aka the Unit Step function) is defined as:}
	$H(t) = 0$ when $t < 0$ and $H(t) = 1$ when $t \ge 0$\\
	\subsubsection*{Examples:}
	eg $e^{-t}H(t) = 0$ for $t < 0$ and $e^{-t}$ for $t \ge 0$\\
	
	We can shift this effect:\\
	$(4 - t^2)H(t-1) = 0 $ for $t < 1$ and $4-t^2$ for $t \ge 1$\\
	
	We can use H(t) to express almost any piecewise-defined fuction\\
	
	ie: Sketch the graph of $f(t) = t^2 + (t-t^2)H(t) + (1-t)H(t-1)$\\
	Solution:\\
	For $t < 0$ $f(t) = t^2 + 0 + 0 = t^2$\\
	For $0 \le t < 1$ $f(t) = t^2 + (t-t^2) + 0 = t$\\
	For $t \ge 1$ $f(t) = t^2 + (t-t^2) + (1-t) = 1$\\
	
	We'll need to do this in reverse:\\
	eg. Write $f(t) = |t|$ in terms of $H(t)$\\
	$|t| = -t$ for $t < 0$ and $t$ for $T \ge 0$\\
	Then: $f(t) = -t + 2tH(t)$\\
	Comment: There are other ways to do this, however our standard strucutre will (always) be:
	$$f(t) = \underline{...} + \underline{...} H(t-a) + \underline{...} H(t-b) + \underline{...} H(t-c) + ...$$
	Exxample: Rewrite:\\
	$f(t) = 2t + 4$ for $t < -1$ and $3$ for $t \epsilon [-1, 1)$ and $3t$ for $t \epsilon [1, \infty)$\\
	
	Solution:\\
	We want $f(t) = \_ + \_H(t+1) + \_H(t-1)$\\
	Filling in the blanks:\\
	$$f(t) = (2t + 4) + (-2t - 1)H(t+1) + (3t - 3)H(t-1)$$\\
	
	In general:\\
	$f(t) = $ $f1$ if $t < a$, $f2$ if $t \epsilon [a,b)$ $f3$ if $t \ge b$\\
	$$f(t) = f1 + (f2 - f1)H(t-a) + (f3-f2)H(t-b)$$
	
	There are a couple of tricks which may be useful:\\
	\begin{itemize}
		\item Consider 1 - H(t-a) this is $1$ for $t < a$ and $0$ for $t \ge a$ (an off switch)
		\item Consider $H(t-a) - H(t-b)$ where $a < b$ = $0$ for $t < a$ and $1$ for $a \le t \le b$ and $0$ for $t \ge b$
	\end{itemize}
	
	\subsection*{The Second Shift Theorem}
	What is $L\{H(t)\}$? It's $L\{H(t)\} = \int_0^{\infty}H(t)e^{-st}dt$\\
	$= \int_0^{\infty}e^{-st}dt = L{1} = \frac{1}{S^2}$ for $s > 0$\\
	
	More generally: $L\{f(t)H(t)\} = L\{f(t)\} = F(s)$\\
	From now on we'll write our fomulas as:\\
	
	\begin{tabular}{c | c}
		$L\{H(t)\} = \frac{1}{S}$ & $L^-1\{\frac{1}{S}\} = H(t)$\\
		$L\{e^{at}H(t)\} = \frac{1}{S-a}$ & $L^{-1}\{\frac{1}{S-a} \} = e^{at}H(t)$\\
		etc & etc 
	\end{tabular}
	
	What happens to H(t-a)?\\
	$L\{H(t-a)\} = \int_0^{\infty}H(t-a)e^{-st}dt$\\
	$ = \int_0^a 0 * e^{-st}dt + \int_a^{\infty} 1 * e^{-st}dt$\\
	$ = 0 - \frac{e^{-st}}{S} \_a^{\infty} = \frac{e^{-as}}{S}$\\
	And now the big question, what about $F(t)H(t-a))$\\
	
	\subsubsection*{Proof}
	$L\{f(t-a)H(t-a)\} = e^{-as}F(s)$\\
	$\therefore L\{f(t-a)H(t-a)\} = \int_0^{\infty}e^{-st}f(t-a)H(t-a)dt$\\
	We let $\tau = t-a$ and $d\tau = dt$\\
	$ = \int_{-a}^{\infty}e^{-s(\tau +a)}f(\tau)H(\tau)d\tau$\\
	$= \int_{-a}^{0}0d\tau + e^{-as}\int_0^{\infty}e^{-s\tau}f(\tau)d\tau$
	$= e^{-as}F(s)$\\
	
	\begin{tabular}{ c | c }
		$L\{H(t)\} = \frac{1}{S}$ & $L^{-1}\{ \frac{1}{S} \} = H(t)$ \\
		$L\{ e^{at}H(t) \} = \frac{1}{S-a}$ & $L^{-1}\{\frac{1}{S-a} \} = e^{at}H(t)$ \\
		$L\{t^n H(t)\} = \frac{n!}{S^{n+1}}$ & $L^{-1}\{\frac{n!}{S^{n-1}} \} = t^n H(t)$ \\
		$L\{ cos atH(t) \} = \frac{S}{S^2 + a^2}$ & $L^{-1}\{\frac{S}{S^2 + a^2} \} = cos at H(t)$\\
		$L\{sin at H(t) \} = \frac{a}{S^2 + a^2}$ & $L^{-1}\{\frac{a}{S^2+a^2} \} = sin at H(t)$\\
		$L\{f(t-a)H(t-a)\} = e^{-as}F(s)$ & $L^{-1}\{ e^{-as}F(s)\} = f(t-a)H(t-a)$\\
		$L\{e^{at} f(t)\} = F(s-a)$ & $L^{-1}\{F(s-a) \} = e^{at}f(t)$ \\
	\end{tabular}
		
	\subsubsection*{Using the 2nd shift theorem}
	Example1)\\
	Let f(t) = t for $t \epsilon [0,2)$ and 0 for $t \ge 2$\\
	Find F(s)\\
	Solution:\\
	Write $f(t) = t[H(t) - H(t-2)] = tH(t) - tH(t-2) = tH(t) - (t-2+2)H(t-2)$
	$ = t * H(t) - (t-2)H(t-2) - 2H(t-2)$\\
	and now we can see that:\\
	$F(s) = \frac{1}{S^2} - \frac{e^{-2s}}{S^2} - \frac{e^{-2s}}{S}$\\
	
	2)\\
	Find G(s) if g(t) = $\frac{1}{3}t$ for $t \epsilon [0,3)$ and $2-\frac{1}{3}t$ for $t \epsilon [3,6)$ and 0 for $t \ge 6$\\
	
	Solution:\\
	$$g(t) = \frac{1}{3}t[H(t) - H(t-3)] + (2-\frac{1}{3}t)[H(t-3) - H(t-6)]$$
	$$=  \frac{1}{3}tH(t) + (2-\frac{2}{3}t)H(t-3) + (\frac{1}{3}t - 2)H(t-6)$$
	$$= \frac{1}{3}tH(t) - \frac{2}{3}(t-3)H(t-3) + \frac{1}{3}(t-6)H(t-6)$$
	$$\rightarrow G(s) = \frac{1}{3S^2} - \frac{2}{3S^2}e^{-3s} + \frac{1}{3S^2}e^{-6s}$$
	
	3)\\
	Evaluate $L^{-1}\{\frac{e^{-S}}{S^2 + 1} \}$\\
	Solution:\\
	Since $L^{-1}\{\frac{1}{S^2  + 1} \} = sint H(t)$\\
	We have $L^{-1}\{\frac{e^{-s}}{S^2 + 1} = sin(t-1)H(t-1) \}$\\
	
	4)\\
	Find $L^{-1}\{\frac{e^{-3S}}{(S+1)^4} \}$\\
	Solution:\\
	$L^{-1}\{\frac{1}{(S+1)^4} \} = e^{-t}L^{-1}\{\frac{1}{S^4} \}$\\
	$ = \frac{1}{6}e^{-t}L^{-1}\{ \frac{6}{S^4} \}$\\
	$ = \frac{1}{6}t^3e^{-t}H(t)$\\
	$\rightarrow L^{-1}\{\frac{e^{-3S}}{(S+1)^4} \} = \frac{1}{6}(t-3)^3e^{-(t-3)}H(t-3)$\\
	
	5)\\
	Solve the IVP $\frac{dy}{dt} + 3y = f(t)$, y(0) = 0\\
	where $f(t) = 3$ for $t \epsilon [0,10)$ and $6$ for $t \epsilon [10, 20)$ and 3 for $t \epsilon [20, 30)$ and 0 for $t \epsilon [30, \infty)$\\
	
	Solution:\\
	$$y' + 3y = 3H(t) + 3H(t-10) - 3H(t-20) - 3H(t-30)$$
	$$\rightarrow sY + 3Y = \frac{3}{S} + \frac{3}{S}e^{-10S} - \frac{3}{S}e^{-20S} - \frac{3}{S}e^{-30S}$$
	$$\rightarrow Y(s) = \frac{3}{S(S+3)}[1 + e^{-10s} - e^{-20S} - e^{-30S}]$$
	$$Y(s) = [\frac{1}{S} - \frac{1}{S+3}](1 + e^{-10S} - e^{-20S} - e^{30S})$$
	$$\rightarrow y(t) = (1-e^{-3t})H(t) + (1 - e^{-3(t-10)})H(t-10) - (1-e^{-3(t-20)})H(t-20) - (1-e^{-3(t-30)})H(t-30)$$
	
	In piecewise-defined form:\\
	$y(t) =$\\
	$1-e^{-3t}$ for $t\epsilon [0, 10)$\\
	$2 - (1 + e^{30})e^{-3t}$ for $ t\epsilon [10. 20)$\\
	$1 - (1 + e^{30} - e^{60})e^{-3t}$ for $t \epsilon [20, 30)$\\
	$- (1 + e^{30} - e^{60} - e^{90}) e^{-3t}$ for $t \epsilon [30, \infty)$\\
	
	6)\\
	Solve the IVP $x'' + 3x' + 2x = f(t)$ x(0) = 1, x'(0) = -1\\
	where $f(t) = = sin(t) H(t-\frac{\pi}{2})$\\
	Solution:\\
	Note that $f(t) = cos(t-\frac{\pi}{2})H(t-\frac{\pi}{2})$\\
	So $(S^2X(s) - sx(0) - x'(0)) - 3(sX(s) - x(0)) + 2X(s)$\\
	$= \frac{S}{S^2 + 1}e^{\frac{-\pi}{2}S}$\\
	$\rightarrow (S^2 + 3S + 2)X(s) = s + 2 + \frac{S}{S^2 + 1}e^{\frac{-\pi}{2}S}$\\
	$$\rightarrow x(s) = \frac{S+2}{(s+1)(s-2)} + \frac{se^{\frac{-\pi S}{2}}}{(S^2 + 1)(s+1)(s+2)}$$
	
	Example:\\
	Solve hte IVP $x'' + 3x' + 2x = f(t)$ $x(0) = 1$, $x'(0) = -1$\\
	Where $f(t) = sint H(t-\frac{\pi}{2})$\\
	
	Solution: First $sin(t) = cos(t - \frac{\pi}{2})$ so $f(t) = cos(t-\frac{\pi}{2})H(t-\frac{\pi}{2})$\\
	
	Applying the transofmration gives:\\
	$$S^2X - sx(0) - x'(0) + 3sx - 3x(0) + 2x = \frac{S}{S^2 + 1}e^{\frac{\pi}{2}S}$$
	$$\rightarrow x(s) = \frac{s+2}{(s+1)(s+2)} + \frac{se^{-\frac{\pi}{2}s}}{(s+1)(s+2)(s^2+1)}$$
	$$= \frac{1}{s+1} + e^{-\frac{\pi}{2}s}[\frac{\frac{-1}{2}}{s+1} + \frac{\frac{2}{5}}{s+2} + \frac{\frac{1}{10}s + \frac{3}{10}}{s^2 + 1}]$$
	
	This gives us:\\
	$$x(t) = e^{-t}H(t) + H(t-\frac{\pi}{2}) + [\frac{-1}{2}e^{-t} + \frac{2}{5}e^{\pi}e^{-2t} + \frac{1}{10}sin(t) - \frac{3}{10}cos(t)]$$
	
	\section*{Systems of First-Order Linear Equations}
	In some applications (mixing tanks, chemical reactions etc) we find that he behaviour of multiple quantities are related to each other. In a relatively simple case, we might have:\\
	\begin{enumerate}
		\item $\frac{dx}{dt} = ax + by + f_1(t)$
		\item $\frac{dy}{dt} = cx + dy + f_2(t)$
	\end{enumerate}
	
	Since these are linear, it is possible to write the system in vector form.\\
	Letting $\overrightarrow{x}(t) =
	\begin{bmatrix}
	x(t) \\ y(t)
	\end{bmatrix}$
	\\ we have:
	$\begin{bmatrix}
	x' \\ y'
	\end{bmatrix}
	=
	\begin{bmatrix}
	a & b \\
	c & d \\
	\end{bmatrix}
	\begin{bmatrix}
	x \\ y
	\end{bmatrix}
	+
	\begin{bmatrix}
	f_1 \\ f_2
	\end{bmatrix}
	$\\
	
	More concisely, $\overrightarrow{x}' = A\overrightarrow{x} + \overrightarrow{f}(t)$\\
	
	Eg:\\
	If $\frac{dx}{dt} = 2x - 3y + t$\\
	and $\frac{dy}{dt} = x + 4y$\\
	Then we would want to write:\\
	$\overrightarrow{x}' = 
	\begin{bmatrix}
	2 & -3 \\ 1 & 4
	\end{bmatrix}
	\overrightarrow{x}
	 +
	 \begin{bmatrix}
	 t \\ 0
	 \end{bmatrix}$
	 
	 Observation:\\
	 Every 1st-order linear 2 variable vector DE can be converted int a 2nd-order originary DE\\
	 
	 How? Notice that we can solve 1 for y:\\
	 $$y = \frac{x'}{b} - \frac{a}{b}x - \frac{f_1(x)}{b}$$
	 
	 Therefore:\\
	 $$\frac{dy}{dt} = \frac{x''}{b} - \frac{a}{b}x' - \frac{f_1'(t)}{b}$$
	 and so 2 becomes (after simplifying):\\
	 $$x'' + (a+d)x' + (ad-bc)x = f_1'(t) - df_1(t) + bf_2(t)$$
	 
	 It is also possible to express ODE's as vector DE's
	 eg:\\
	 Suppose $y'' + py' + qy = 0$ with $y(0) = y_0$ and $y'(0) = V_0$\\
	 
	 If we define $\overrightarrow{x}(t) = \begin{bmatrix}
	 y \\ y'
	 \end{bmatrix}$
	 Then $\overrightarrow{x}' = \begin{bmatrix}
	 y' \\ y''
	 \end{bmatrix} = \begin{bmatrix}
	 y' \\ -py' - qy
	 \end{bmatrix} = \begin{bmatrix}
	 0 & 1 \\ -p & -q
	 \end{bmatrix} \begin{bmatrix}
	 y\\ y'
	 \end{bmatrix}$
	 
	 This tesllls us that the theory of ODE's can be extended to vector DE's!\\
	 \begin{itemize}
	 	\item The IVP $\overrightarrow{x}' = A\overrightarrow{x} + \overrightarrow{f}(t)$ with $\overrightarrow{x}(0) = \overrightarrow{x}_0$
	 	\item a general solution will contain two arbitrary constants (and 2 linearly independent functions)
	 	\item The Principle of superposition applies:\\
	 	$\rightarrow$ we can adopt hte same basic stragey to solve $\overrightarrow{x}' = A\overrightarrow{x} + \overrightarrow{f}(x)$ we can proceed by solving $\overrightarrow{x_h} = A\overrightarrow{x_h}$ and then finding a particular solution $\overrightarrow{x_p}$ the gneeral solution will be $\overrightarrow{x} = \overrightarrow{x_h} + \overrightarrow{x_p}$
	 \end{itemize}
\end{document}
