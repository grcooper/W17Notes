\documentclass[12pt]{article}

\setlength\parindent{0pt}
\newcommand{\myt}[1]{\textbf{\underline{#1}}}

\usepackage{mathtools}
\usepackage{amssymb}
\usepackage{graphicx}

\title{\vspace{-15ex}Lecture 10\vspace{-1ex}}
\date{February 8th, 2017}
\author{Graham Cooper}

\begin{document}
	\maketitle
	
	\section*{Oscillatory Systems}
	(Applications of 2nd-order linear DE's)
	If \underline{MechanicalOscillators}
	\begin{itemize}
		\item simple pendulum (for small oscillations at least)
		\item a bouy or boat (floating object in water)
		\item cantilevers (something fixed at one end ie a building oscillating in the wind)
	\end{itemize}
	
	\subsection*{Spring(mechanical) Oscillator}
	As a "thought experiment", consider a mass attached to a spring, sliding on a frictionless surface. We will assume that the spring obey's Hooke's Law:\\
	Force from spring = $F_{spring} = -kx$\\s
	
	We will also consider the effect of a damping force\\
	$f_{damping} = -c\frac{dx}{dt}$ for $c > 0$\\
	If we consider no other forces, then Newton's 2nd law of motion gives:\\
	$F = ma$
	$\rightarrow -kx - c\frac{dx}{dt} = m\frac{d^2x}{dt}$\\
	
	That is, $x''(t) + \frac{c}{m}x'(t) + \frac{k}{m}x = 0$\\
	
	We can also consider the effect of an "external" force F(t)\\
	$$m\frac{d^2x}{dt^2} = -kx-c\frac{dx}{dt} + F(t)$$
	$$\rightarrow x'' + \frac{c}{m}x' + \frac{k}{m}x = \frac{F(t)}{m}$$
	Traditionally this is rewritten in terms of:\\\\
	$\omega_0 = \sqrt{frac{k}{m}}$ The natural frequency\\
	and:\\
	$\zeta = \frac{c}{2\omega_0m} = \frac{c}{2\sqrt{km}}$ (the damping parameter)\\
	Note: $\zeta$ is dimensionless!\\
	
	$f(t) = \frac{F(t)}{m}$\\
	With these, we have
	$$x''(t) + 2\omega_0\zeta x'(t) + \omega_0^2x(t) = f(t)$$
	
	\subsection*{The Electrical Oscillator}
	Consider a circuit consisting of a source voltage, a resistor, a capacitor and an inductor in series\\
	(see fig10.1)
	A variation of Faraday's law states that the voltage drop across the inductor is proportional to the derivative of the current.\\
	$V_L(t) = L\frac{di}{dt}$ (L is called the inductance)\\
	Combining this with our other laws we have:\\
	$V(t) = V_R(t) + V_C(t) + V_L(t)$\\
	$= Ri + \frac{1}{c}q + L\frac{di}{dt}$\\
	
	So for the current we have:\\
	$V'(t) = R\frac{di}{dt} + \frac{i}{c} + L\frac{d^2i}{dt^2}$\\
	While for the charge we have:\\
	$V(t) = R\frac{dq}{dt} + \frac{1}{c}q + L\frac{d^2q}{dt}$\\
	
	Now if we let $\omega_0 = \frac{1}{\sqrt{LC}}$, $\zeta = \frac{R}{2\omega_0L}$, $f(t) = \frac{V(t)}{L}$\\
	We get:\\
	$$q''(t) + 2\omega_0\zeta q'(t) + \omega_0^2 q(t) = f(t)$$
	This is the same DE as fo the mechanical oscillator! We can draw an analogy\\
	
	Compare:\\
	$$mx'' + cx' + kx = F(t)$$
	$$Lq'' + Rq' + \frac{1}{c}q = V(t)$$
	
	We can see that displacement corresponds to charge ( quantity of interest)\\
	We can see that applied force corresponds to applied voltage (sources of energy)\\
	we can see that spring corresponds to capacitor (storage of energy)\\
	We can see that damper corresponds to resistor (appose the motion within the system/lose energy))\\
	we can see that mass corresponds to inductance ( momentum )\\
	
	\subsection*{Free Oscillations}
	With no external force/voltage we have:\\
	$$y'' + 2\omega_0\zeta y' + \omega_0^2 y = 0$$
	The characteristic equation is:\\
	$$m^2 + 2\omega_0\zeta m + \omega^2 = 0$$
	The roots are:\\
	$$m = \frac{-2\omega_0 \zeta \pm \sqrt{4\omega_0^2\zeta^2 - 4\omega_0^2}}{2}$$
	$$m = -\omega_0\zeta \pm \omega_0\sqrt{\zeta^2 - 1}$$
	Now we can see the reason for using $\zeta$. The critical point between real and complex roots is $\zeta = 1$ ie. $C = 2\sqrt{km}$ or $R = L^{\frac{3}{2}}\sqrt{c}$
	
	\subsubsection*{Case 1a $\zeta = 0$ (No damping at all)}
	
	Here we have:\\
	$$y'' + \omega^2 y = 0$$
	so $y = C_1cos(\omega_0 t) + C_2sin(\omega_0 t)$\\
	now we see the reason for calling $\omega_0$ the natural frequency:\\
	If $y(0) = 0$ and $y'(0) = 0$ then $0 = c_1$ and $C_2\omega_0 = 0$ so $C_1 = C_2 = 0$ and $y = 0$\\
	
	If $y(0) = y_0$ and $y'(0) = 0$ (initial displacement), we get:\\
	$y_0 = C_1$ and $0 = C_2$, so $y = y_0cos(\omega_0 t)$\\
	We call this behaviour simple harmonic motion (SHM)\\
	If $y(0) = 0$ and $y'(0) = v_0$ (initial velocity), we get:\\
	$0 = C_1$ and $V_0 = C_2\omega_0$ so $C_2 = \frac{V_0}{\omega_0}$, and so $y = \frac{V_0}{\omega_0}sin(\omega_0 t)$\\
	
	If $y_0 = y_0$ and $y'(0) = v_0$ then $y_0 = C_1$ and $v_0 = C_2\omega_0$\\
	$$y = y_0cos(\omega_0 t) + \frac{V_0}{\omega_0}sin(\omega_0 t)$$
	This is still SHM; it can be expressed asa single cosine or sine function\\
	
	How? Work backwards:\\
	$$cos(A + B) = cos(A)cos(B) - sin(A)sin(B)$$
	$$\rightarrow Rcos(\omega t - \phi) = R[cos(\omega_0 t)cos(\phi) + sin(\omega_0 t)sin(\phi)]$$
	$$= (Rcos(\phi))cos(\omega_0 t) + (Rsin(\phi)sin(\omega_0 t)$$
	Now ewe set: $Rcos(\phi) = C_1$ and $Rsin(\phi) = C_2$\\
	Then: $R = \sqrt{C_1^2 + c_2^2}$ and $tan(\phi) = \frac{C_2}{C_1}$\\
	
	So our general solution $y = C_1cos(\omega_0 t) + C_2sin(\omega_0 t)$ can also be expressed as $y = Rcos(\omega_0 t - \phi)$\\
	
	Expressed as $y = Rcos(\omega_0 t - \phi)$\\
	With this, $y= \sqrt{y_0^2 + \frac{v_0^2}{\omega_0^2}}cos(\omega_0 t - \phi)$ where $tan(\phi) = \frac{v_0}{y_0\omega_0}$
	
	\subsubsection*{Case 1b $zeta \epsilon (0,1)$}
	In this case we have $m = -\omega_0 \zeta \pm \omega \sqrt{1 - \zeta^2}i$\\
	(We are using: $\pm\sqrt{-a} = \pm ai$)
	so $y = e^{-\omega_0\zeta t}[C_1 cos(\omega_0 \sqrt{1-\zeta^2}t) + C_2sin(\omega\sqrt{-\zeta^2}t)]$\\
	
	Or we can write $y = Re^{-\omega_0 \zeta t}cos(\omega_0 \sqrt{1 - \zeta^2}t - \phi)$\\
	
	The values of R and $\phi$ will depend on $y(0)$ and $y'(0)$ bein in general we get \underline{overdamped} behaviour.\\
	
	Note: The frequency is $\omega_0\sqrt{1 - \zeta^2}i$ as $\zeta$ increases the oscillations slow down!\\
	
	\subsubsection*{Case 2 $\zeta \epsilon (1, \infty)$}
	Here we get $m = -\omega_0\zeta \pm \omega_0 \sqrt{\zeta^2 - 1}$ and $y = C_1 e^{-\omega_0(\zeta + \sqrt{\zeta^2 - 1})t} + C_2e^{-\omega_0(\zeta-\sqrt{\zeta^2 - 1})t}$
	
	We call this overdamped motion\\
	
	\subsubsection*{Case 4 $\zeta = 1$}
	This is critical damping. The graphs look very similar to the overdamped case\\
	
	\subsection*{Forced Oscillations}
	
	\subsubsection*{Simple Case:}
	 A constant force. With a constant force applied to the mass. We have:\\
	
	$$x'' + 2\omega_0 \zeta x' + \omega_0^2 x = f_0$$
	The solution will be: $x_h + x_p$ where $x_h$ is as we disucssed last time and $x_p = \frac{f_0}{\omega_0^2}$
	
	Guess: $x_p = A$, $x_p' = 0$, $x_p'' = 0$ and so:\\
	$0 + 0 + \omega_0^2A = f_0$ so $A = \frac{f_0}{\omega_0^2}$\\
	Interpolation? The equilibirum position moves by a distance:\\
	$\frac{f_0}{\omega_0^2}$
	
	\subsubsection*{Harder case}
	Oscillatory forcing:\\
	Suppose:\\
	$$x'' + 2\omega_0 \zeta x' + \omega_0^2 x = f_0cos(\omega_0t)$$
	
	A recap might be helpful before finishing the nondimensionalization. For a mechanical oscillator we consider:\\
	$$mx'' + cx' + kx = F_0cos(\omega t)$$
	Defining $\omega_0 = \sqrt{\frac{k}{m}}$ and $\zeta = \frac{c}{2\omega_0 m}$\\
	Letting y = x, we get $y'' + 2\omega_0 \zeta y' + \omega_0^2 y = f_0 cos(\omega t)$\\
	
	For an electrical Oscillator:\\
	$$Lq'' + Rq' + \frac{1}{C}q = V_0 cos(\omega t)$$
	Defining $\omega_0 = \frac{1}{\sqrt{LC}}$, $\zeta = \frac{R}{2\omega_0 L}$, and $f_0 = \frac{V_0}{L}$\\
	Leting y = q we get the above\\
	
	Now we have:\\
	\begin{itemize}
		\item $[y] = L or Q$
		\item $[t] = T$
	\end{itemize}
	and constants:\\
	\begin{itemize}
		\item $[\omega_0] T^{-1}$
		\item $[\omega] = T^{-1}$
		\item $[f_0] = LT^{-2}$ or $QT^{-2}$
	\end{itemize}

	Lets let $t_c = \frac{1}{\omega_0}$ and $y_c = \frac{f_0}{\omega_0^2}$\\
	Define $\tau = \frac{t}{t_c} = \omega_0 t$ and $y = \frac{y}{y_c} = \frac{\omega_0^2 y}{f_0}$\\
	
	With these:\\
	$\frac{dy}{dt} = \frac{dy}{dY} \frac{DY}{d\tau}\frac{d\tau}{dt} = (\frac{f_0}{\omega_0^2})\frac{dY}{d\tau}(\omega_0) = \frac{f_0}{\omega_0}\frac{dY}{d\tau}$\\
	And:\\
	$\frac{d^2y}{dt^2} = \frac{d}{dt}(\frac{dy}{dt}) = \frac{d}{dt}(\frac{f_0}{\omega_0}\frac{dY}{d\tau})$\\
	$= \frac{f_0}{\omega_0}\frac{d}{d\tau}(\frac{dY}{d\tau})\frac{d\tau}{dt}$\\
	$= (\frac{f_0}{\omega_0})\frac{d^2Y}{d\tau^2}(\omega_0) = f_0\frac{d^2Y}{d\tau^2}$\\
	
	Our DE becomes:\\
	$$f_0\frac{d^2Y}{d\tau^2} + 2\omega_0\zeta(\frac{f_0}{\omega_0})\frac{dY}{d\tau} + \omega_0^2(\frac{f_0Y}{\omega_0^2}) = f_0 cos(\omega(\frac{\tau}{\omega_0}))$$
	That is, :\\
	$$\frac{d^2Y}{d\tau^2} + 2\zeta \frac{dY}{d\tau} + y = cos(\Omega\tau)$$
	Where $\Omega = \frac{\omega}{\omega_0}$\\
	
	\subsubsection*{Finding a particular solution}
	Can simply guess $Y_p = Acos(\Omega\tau) + Bsin(\Omega\tau)$ (Try it!)\\
	Alternatively we can use complex numbers as a shortcut:\\
	We take our equation: $y'' 2\zeta y' + y = cos(\Omega\tau)$\\
	And add: $i\Xi'' + 2\zeta i \xi' + i\xi = isin(\Omega\tau)$
	We get: $Z'' + 2\zeta z' + z = e^{i\Omega\tau}$\\
	Where $y = i \xi = z$\\
	
	If we can find $z_p$ then $y_p = R_e(z_p)$\\
	
	For $z_p$ we guess $z = \alpha e^{i\Omega\tau}$ where $\alpha \epsilon Complex$\\
	The $z' = i\alpha \Omega e^{i\Omega\tau}$\\
	And $z'' = -\alpha\Omega^2e^{i\Omega\tau}$\\
	
	Plug into the DE:\\
	$$-\alpha\Omega^2e^{i\Omega\tau} + 2i\alpha\Omega\zeta e^{i\Omega\tau} + \alpha e^{i\Omega\tau} = e^{i\Omega\tau}$$
	$$\rightarrow \alpha[1 - \Omega^2] + 2i \alpha \Omega \zeta = 1$$
	$$\rightarrow \alpha = \frac{1}{[1-\Omega^2] + 2i\Omega\zeta}$$
	$$ = \frac{1}{\sqrt{(1 - \Omega^2)^2 + 4\Omega\zeta^2}e^{i\delta}}$$
	Where $tan(\delta) = \frac{2\Omega\zeta}{1 - \Omega^2}$\\
	$$= \frac{e^{-i\delta}}{\sqrt{(1-\Omega^2)^2 + 4\Omega^2\zeta^2}}$$
	We can conlude that:\\
	$z_p = \alpha e^{i\Omega\tau} = Ae^{-i\delta}{e^{i\Omega\tau}}$\\
	$= Ae^{i(\Omega\tau - \delta)}$\\
	Therefore $Y_p = Acos(\Omega\tau - \delta)$\\
	Where $A = \frac{1}{\sqrt{(1-\Omega^2)^2 + 4\zeta^2\Omega^2}}$ and $tan(\delta) = \frac{2\Omega\zeta}{1-\Omega^2}$\\
	
	We'll call $y_h$ the \underline{transient} part of the solution, since $y_h \rightarrow 0$ as $\tau \rightarrow \infty$ (if $\zeta > 0$)\\
	We call $y_p$ the \underline{steady-state} solution\\
	
	\subsubsection*{Analysis of the steady-state}
	\begin{enumerate}
		\item The Aplituted response function: It will help to think of $\zeta$ as fixed and $\Omega$ as being a variable. $A = A(\Omega)$
		We can see that:
		\begin{enumerate}
			\item $A(\Omega) \rightarrow 1$ as $\Omega \rightarrow 0$ for any $\zeta$
			\item $A(\Omega) \rightarrow 0$ as $\Omega \rightarrow \infty$ for any $\zeta$
			\item If $\zeta = 0$, $A = \frac{1}{\sqrt{(1-\Omega^2)^2}} = \frac{1}{|l - \Omega^2|}$
			\item We can show that $A'(\Omega) = 0$ when $\Omega = \sqrt{1-2\zeta^2}$ (if $\zeta < \frac{1}{\sqrt{2}})$
		\end{enumerate}
		\item The Phase Response Function, $\delta(\Omega)$\\
		We have $tan(\delta) = \frac{2\Omega\zeta}{1-\Omega^2}$\\
		In order to make sure that $\delta > 0$ we have to let:\\
		$\delta = tan^{-2}(\frac{2\Omega\zeta}{1-\Omega^2})$ for $\Omega \epsilon (0,1)$\\
		$\delta = \frac{\pi}{2}$ for $\Omega = 1$\\
		$\delta = tan^{-1}(\frac{2\Omega\zeta}{1-\Omega^2}) + \pi$ for $\Omega \epsilon (1, \infty)$
	\end{enumerate}
	
	
	
	
\end{document}
