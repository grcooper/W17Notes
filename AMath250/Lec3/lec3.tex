\documentclass[12pt]{article}

\setlength\parindent{0pt}
\newcommand{\myt}[1]{\textbf{\underline{#1}}}

\usepackage{mathtools}
\usepackage{amssymb}
\usepackage{graphicx}

\title{\vspace{-15ex}Amath 250 Lecture 3\vspace{-1ex}}
\date{January 9th, 2017}
\author{Graham Cooper}

\begin{document}
	\maketitle
	
	\section*{The Existence/Uniqueness theorem (for 1st order IVPs)}
	The IVP:
	$$\frac{dy}{dx} = f(x,y), y(x_0) = y_0$$
	Has a unique solutino provided that both f(x,y) and $f_y(x,y)$ are continuous within a neighbourhood of $(x_0, y_0)$\\
	
	\myt{Corollary}: If f and $f_y$ are cts in a region of the xy-plane, then the various solution curves of $y'=f(x,y)$ will not intersect in that region.\\
	
	\subsection*{Sketching Families of Solutions}
	
	If you have problems that are too difficult - you can look at the derivitive. Using the DE we already have info about the derivitive and we can use that info directly.\\
	
	We can often tell a lot about he graphs of the solutions from the DE itself.\\
	
	Example 1: Consider the equation:\\
	$\frac{dy}{dx} = y^2 - 4$\\
	
	If we set the $\frac{dy}{dx} = 0$ we get $y = +/- 2$\\
	This means that:
	\begin{itemize}
		\item the constant functions $y = +/-2$ are solutions (They are called "equilibrium solutions)
		\item none of the other solutions have any critical points
		\item $\frac{dy}{dx} > 0$ iff $|y| > 2$ and $\frac{dy}{dx} < 0$ iff $|y| < 2$
		\item in fact, $\frac{dy}{dx} \ne 0$ when $y \approx +/-2$, and $|\frac{dy}{dx}|$ increases with distance from $y = +/-2$
	\end{itemize}

	Inflection points?
	
	If $\frac{dy}{dx} = y^2 - 4$ then $\frac{d^2y}{dx^2} = 2y\frac{dy}{dx} = 2y(y^2 - 4)$\\
	There is a line of inlection points along the x-axis\\
	
	Of course we \underline{can} solve this DE.\\
	
	$$\frac{dy}{dx} = y^2 - 4$$
	$$\int \frac{dy}{y^2-4} = \int dx$$
	$$\frac{1}{4} \int (\frac{1}{y-2} - \frac{1}{y+2})dy = x + c_1$$
	$$ln|y-2| - ln|y+2| = 4x + 4c_1$$
	$$ln|\frac{y-2}{y+2} = 4x + c_2$$
	$$|\frac{y-2}{y+2}| = e^{4x+c_2}$$
	$$= e^{c_2}e^{4x}$$
	$$\frac{y-2}{y+2} = c_3e^{4x}$$
	$$y-2 = c_3(y+2)e^{4x}$$
	$$y(1-c_3e^{4x}) = 2 + 2c_3e^{4x}$$
	$$y = 2(\frac{1 + Ce^{4x}}{1-Ce^{4x}})$$
	
	Partial fraction decomposition for above:\\
	$$\frac{1}{y^2-4} = \frac{A}{y+2} + \frac{B}{y-2}$$
	$$1 = A(y-2) + B(y+2)$$
	Setting $y=2$ gives $B = \frac{1}{4}$\\
	Setting $y=-2$ gives $A = -\frac{1}{4}$\\
	
	The solution will have verticle asymptotes when:\\
	$$1-Ce^{4x} = 0$$
	ie $x = \frac{1}{4}ln(\frac{1}{c}) = -\frac{1}{4}lnC$\\
	(see figure 3.1)
	
	\subsection*{Guidelines for sketching}
	\begin{itemize}
		\item solve the DE if possible
		\item Identify any "exceptional" solutions. These may be singular solutions or equilibrium solutions, or other. for linear equations, setting $C = 0$ will usually yield an exceptional solution
		\item Consider the behaviour of other solutions as $x \rightarrow \pm \infty$ or near vertical asymptotes
		\item For more detail, find out where $y' = 0$. This will give you a curve, called the horizontal isocline, on which every solution to the DE has a horizontal tangent
		\item (optional: do the same with $y''$)
	\end{itemize}

	\myt{Example:}\\
	Consider the DE $\frac{dx}{dy} = y - x^2$\\
	In standard form: $y' - y = -x^2$\\
	Int. Factor: $I(x) = e^{\int K(x)dx} = e^{-\int dx}$\\
	Factor this in:\\
	$$e^{-x}\frac{dy}{dx} - ye^{-x} = -x^2e6{-x}$$
	$$\frac{d}{dx}ye^{-x} = -x^2e^{-x}$$
	$$ye^{-x} = -\int x^2e^{-x}dx$$
	$$= ... = e^{-x}(x^2 + 2x+2) + C$$
	Need to take double integral above, but we skip that step\\
	$$y = x^2 2x + 2 + Ce^x$$
	Exceptional solutions? $C = 0$ gives a parabola: $y = x^2 + 2x + 2 = (x+1)^2 + 1$\\
	
	(See figure 3.2)\\
	
	How do the other solutions relate to this?\\
	Well, $Ce^x \rightarrow 0$ as $x \rightarrow -\infty$, so every solution approaches the parabola asymptotically as $x \rightarrow -\infty$.\\
	As $x \rightarrow \infty$, $Ce^x \rightarrow \pm \infty$ (Depending on the sign of C), so the olutions diverge from the parabola\\
	
	Critical points? (horizontal isocline?)\\
	$$\frac{dy}{dx} = 0 \rightarrow y = x^2$$
	
	(see figure 3.3) the red is the isocline and that is where points will change direction/where there will be critical points.\\
	
	Another Sketching Example:\\
	Consider $\frac{dy}{dx} = (xy)^{\frac{2}{3}}$\\
	Solve? This is seperable:\\
	$$\frac{dy}{y^{\frac{2}{3}}} = x^{\frac{2}{3}}dx$$
	$$\rightarrow \int y^{\frac{-2}{3}}dy = \int x^{\frac{2}{3}}dx$$
	$$\rightarrow 3y^{\frac{1}{3}} = \frac{3}{5}x^{\frac{5}{3}} + c_1$$
	$$\rightarrow y = (\frac{1}{5}x^{\frac{5}{3}} + c)^3$$
		
	Exceptional solutions? $C = 0$ gives $y = \frac{1}{125}x^5$\\
	$y = 0$ is also a solution (it's singular)\\
	See figure 3.4\\
	Notice that the two lines overlap.\\
	$f(x,y) = (xy)^{\frac{2}{3}}$ is the cts on $\Re^2$, but $f_y(x,y) = \frac{2}{3}x^{\frac{2}{3}}y^{\frac{1}{3}} = \frac{2x^{\frac{2}{3}}}{3y^{\frac{1}{3}}}$ is not cts when $y = 0$\\
	
	Other solutions $(C \ne 0)$ should start to resemble the $c = 0$ solution as $x \rightarrow \pm \infty$. Isocline?\\
	$\frac{dy}{dx} = 0 \iff x = 0$ or $y = 0$\\
	
	
	
	
\end{document}
