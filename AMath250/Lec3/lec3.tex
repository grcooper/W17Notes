\documentclass[12pt]{article}

\setlength\parindent{0pt}
\newcommand{\myt}[1]{\textbf{\underline{#1}}}

\usepackage{mathtools}
\usepackage{amssymb}
\usepackage{graphicx}

\title{\vspace{-15ex}Amath 250 Lecture 3\vspace{-1ex}}
\date{January 9th, 2017}
\author{Graham Cooper}

\begin{document}
	\maketitle
	
	\section*{The Existence/Uniqueness theorem (for 1st order IVPs)}
	The IVP:
	$$\frac{dy}{dx} = f(x,y), y(x_0) = y_0$$
	Has a unique solutino provided that both f(x,y) and $f_y(x,y)$ are continuous within a neighbourhood of $(x_0, y_0)$\\
	
	\myt{Corollary}: If f and $f_y$ are cts in a region of the xy-plane, then the various solution curves of $y'=f(x,y)$ will not intersect in that region.\\
	
	\subsection*{Sketching Families of Solutions}
	
	If you have problems that are too difficult - you can look at the derivitive. Using the DE we already have info about the derivitive and we can use that info directly.\\
	
	We can often tell a lot about he graphs of the solutions from the DE itself.\\
	
	Example 1: Consider the equation:\\
	$\frac{dy}{dx} = y^2 - 4$\\
	
	If we set the $\frac{dy}{dx} = 0$ we get $y = +/- 2$\\
	This means that:
	\begin{itemize}
		\item the constant functions $y = +/-2$ are solutions (They are called "equilibrium solutions)
		\item none of the other solutions have any critical points
		\item $\frac{dy}{dx} > 0$ iff $|y| > 2$ and $\frac{dy}{dx} < 0$ iff $|y| < 2$
		\item in fact, $\frac{dy}{dx} \ne 0$ when $y \approx +/-2$, and $|\frac{dy}{dx}|$ increases with distance from $y = +/-2$
	\end{itemize}


	
	
	
\end{document}
