\documentclass[12pt]{article}

\setlength\parindent{0pt}
\newcommand{\myt}[1]{\textbf{\underline{#1}}}

\usepackage{mathtools}
\usepackage{amssymb}
\usepackage{graphicx}

\title{\vspace{-15ex}Amath 250 Lecture 1\vspace{-1ex}}
\date{January 4th, 2017}
\author{Graham Cooper}

\begin{document}
	\maketitle
	
	\myt{Intro to Differential Equations (DEs)}\\\\
	A DE is an equation relating a function to its own derivative(s).\\
	\underline{Solving} it means identifying the function(s).\\
	Eg. Solve:\\
	$$ \frac{dy}{dx} = y$$
	We can guess the solution:\\\\
	$$y = e^y$$
	$$y = Ke^x | \exists K \epsilon R$$
	(This gives the complete set)\\
	
	Eg2:\\
	$$ y' = xy $$
	Solution:\\
	$$ y = Ce^{\frac{1}{2}x^2}$$
	
	Terminology:\\
	We call the family of solutions the \underline{general solution}. A single solution may be called a \underline{particular solution}\\
	
	To determine the value of C required to find a particular solution we need an \underline{initial condition}. The value of f(x) at some value of x. A DE with an IC is called an \underline{initial value problem (IVP)}\\
	
	Eg1: Solve\\
	$$y' = y^2, y(0) = 1$$
	
	Possible Solution:\\
	$$y = \frac{-1}{x}$$
	The General solution is:\\
	$$y = \frac{-1}{x + c}$$
	If we set $y(0) = 1$ then $1 = \frac{-1}{c}$ so $c = -1$. The particular solution is:\\
	$$y = \frac{1}{1-x}$$
	
	The \underline{order} of a DE is the order of the highest derivative. \\
	Eg1 $(y''')^2 = y-1$ is 3rd-order\\
	Eg2 $\frac{d^2y}{dx^2} + 4\frac{dy}{dx} + 4y = x$ is a 2nd order equation\\
	We will soon be able to solve this, the solution is:\\
	$$y = C_1e^{-2x} + C_2xe^{-2x} + \frac{1}{4}(x-1)$$
	
	We can \underline{verify} this:\\
	$$y' = -2C_1e^{-2x} + C_2e^{-2x}-2C_2xe^{-2x} + \frac{1}{4}$$
	$$y'' = 4C_1e^{-2x} - 2C_2e^{-2x} - 2C_2e^{-2x} + 4C_2xe^{-2x}$$
	$$y'' + 4y' + 4y = 4C_1e^{-2x}-4C_2e^{-2x} + 4C_2xe^{-2x} +$$ 
	$$ (-8C_1e^{-2x} +  4C_2e^{-2x} - 8C_2xe^{-2x} + 1) + (4C_1e^{-2x} + 4C_2xe^{-2x} + x - 1) = x$$
	
	Cancel out the terms and that is how we just get x\\
	
	
	
	
\end{document}
