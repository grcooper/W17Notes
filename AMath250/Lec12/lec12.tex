\documentclass[12pt]{article}

\setlength\parindent{0pt}
\newcommand{\myt}[1]{\textbf{\underline{#1}}}

\usepackage{mathtools}
\usepackage{amssymb}
\usepackage{graphicx}

\title{\vspace{-15ex}Lecture\vspace{-1ex}}
\date{March 13th, 2017}
\author{Graham Cooper}

\begin{document}
	\maketitle
	
	Missed a class in here...
	
	\section*{Solving Homogeneous Vector DEs Continued}
	
	\subsection*{Case3: Repeated Eigenvalues}
	Example:\\
	$\overrightarrow{x}' = \begin{bmatrix}
	4 & 1 \\ -1 & 2
	\end{bmatrix} \overrightarrow{x}$\\
	
	$det(A - \lambda I) = det(\begin{bmatrix}
	4 - \lambda & 1 \\ -1 & 2 - \lambda
	\end{bmatrix})$\\
	$= (4-\lambda)(2-\lambda) + 1$\\
	$= \lambda^2 - 6\lambda + 9 = (\lambda - 3)^2$\\
	So $\lambda = 3$\\
	
	Setting $(A - \lambda I)\overrightarrow{V} = \overrightarrow{0}$ gives\\
	$\begin{bmatrix}
	1 & 1 \\ -1 & -1
	\end{bmatrix}
	\begin{bmatrix}
	v_1\\v_2
	\end{bmatrix} = \begin{bmatrix}
	0 \\ 0
	\end{bmatrix}$\\\\
	
	So $v_1 + v_2 = 0$ and we may use $\overrightarrow{v} = \begin{bmatrix}
	1 \\ -1
	\end{bmatrix}$\\
	
	$\rightarrow$ One solution is $\overrightarrow{x_1} = \begin{bmatrix} 1 \\ -1\end{bmatrix} e^{3t}$\\
	
	What's a second one? Try multipying by t?\\
	
	IF $\overrightarrow{x_1} = \overrightarrow{v}e^{\lambda t}$ then we guess:\\
	$\overrightarrow{x_2} = \overrightarrow{v}te^{\lambda t}$?\\
	This gives $\overrightarrow{x_2}' = \overrightarrow{v}e^{\lambda t} + \lambda \overrightarrow{v}te^{\lambda t}$\\
	ie. $\overrightarrow{v}e^{\lambda t} = \overrightarrow{0}$ which is not true!\\
	
	What else could we try?\\
	
	Consider: we have guessed: $\overrightarrow{x_2} = \overrightarrow{v}te^{\lambda t} = \begin{bmatrix}
	v_1te^{\lambda t} \\ v_2te^{\lambda t}
	\end{bmatrix}$\\
	So:\\
	$\overrightarrow{x} = c_1\overrightarrow{x_1} + c_2\overrightarrow{x_2} = \begin{bmatrix}
	c_1v_1e^{\lambda t} + c_2v_1te^{\lambda t} \\ c_1v_2e^{\lambda t} + c_2v_2te^{\lambda t}
	\end{bmatrix}$\\
	
	We \underline{want}\\
	$\overrightarrow{x} = \begin{bmatrix}
	\alpha_1 e^{\lambda t} + \alpha_2te^{\lambda t} \\
	\alpha_3 e^{\lambda t} + \alpha_4 te^{\lambda t}
	\end{bmatrix}$\\
	Where $\alpha_{1-4}$ are related in some way so that there are only two arbitrary constants\\
	
	Our guess is too restrictive - it requires a particular relationship between fourc oncstants, one wayt to relax this restriction is this:\\
	For $\overrightarrow{x_2}$, we guess $\overrightarrow{x_2} = \overrightarrow{v}te^{\lambda t} + \overrightarrow{u}e^{\lambda t}$ where $\overrightarrow{u}$ is a vector to be determined\\
	
	(This way $\overrightarrow{x}(t) = c_1\overrightarrow{x_1} + c_2\overrightarrow{x_2} = \begin{bmatrix}
	(c_1v_1 + c_2u_1)e^{\lambda t} + c_2v_1te^{\lambda t} \\
	(c_1v_2 + c_2u_2)e^{\lambda t} + v_2v_2te^{\lambda t}
	\end{bmatrix}$\\
	What is $\overrightarrow{u}$ ?\\
	
	Plug $\overrightarrow{x_2}$ into the DE: $\overrightarrow{x}' = A\overrightarrow{x}$ becomes \\
	$\overrightarrow{v}e^{\lambda t} + \lambda\overrightarrow{v}te^{\lambda t} = A\overrightarrow{v}te^{\lambda t} + A\overrightarrow{u}e^{\lambda t}$\\
	$\rightarrow A\overrightarrow{u} - \lambda\overrightarrow{u} = \overrightarrow{v}$\\
	$\rightarrow (A-\lambda I)\overrightarrow(u) = \overrightarrow{v}$\\
	
	We can solve this for $\overrightarrow{u}$ In fact, $\overrightarrow{u}$ is called a generalized eigenvector. Note that $(A-\lambda I)^2 \overrightarrow{u} = (A-\lambda I)\overrightarrow{v} = \overrightarrow{0}$\\
	Conclusion: The guess $\overrightarrow{x_2} = \overrightarrow{v}te^{\lambda t} + \overrightarrow{u}e^{\lambda t}$ works for $\overrightarrow{u}$ is a generalized eigenvector (of order 2)\\
	
	Back to the example....\\
	
	We had $A = \begin{bmatrix}
	4 & 1 \\ -1 & 2
	\end{bmatrix}$\\
	$(A - \lambda I) = \begin{bmatrix}
	1 & 1 \\ -1 & -1
	\end{bmatrix}$\\
	
	To find $\overrightarrow{u}$ solve $(A 0 \lambda I)\overrightarrow{u} = \overrightarrow{v}$\\
	ie: \\
	$\begin{bmatrix}
	1 & 1 \\ -1 & -1
	\end{bmatrix}
	\begin{bmatrix}
	u_1 \\ u_2
	\end{bmatrix} =
	\begin{bmatrix}
	1 \\ -1
	\end{bmatrix}$\\
	$u_1 + u_2 = 1$\\
	
	Setting $u_2 = 0$ gives $u_1 = 1$ so $\overrightarrow{u} = \begin{bmatrix}
	1 \\ 0
	\end{bmatrix}$
	So...
	$\overrightarrow{x_1} = \overrightarrow{v}e^{\lambda t}$\\
	$\overrightarrow{x_2} = \overrightarrow{v}te^{\lambda t} + \overrightarrow{u}e^{\lambda t}$\\
	
	$\rightarrow$ the general solution is:\\
	$$\overrightarrow{x} = c_1\overrightarrow{x_1} + c_2\overrightarrow{x_2}$$
	$$= c_1\overrightarrow{v}e^{\lambda t} + c_2[\overrightarrow{v}te^{\lambda t} + \overrightarrow{u}e^{\lambda t}]$$
	$$= c_1\begin{bmatrix}
	1 \\ -1
	\end{bmatrix} e^{3t} + c_2 [ \begin{bmatrix}
	1 \\ -1
	\end{bmatrix} te^{3 t} + \begin{bmatrix} 1 \\ 0 \end{bmatrix} e^{3t}$$
	$$ = \begin{bmatrix}
	(c_1 + c_2)e^{3t} + c_2te^{3t} \\-c_1e^{3t} - c_2te^{3t}
	\end{bmatrix}$$
	
\end{document}
