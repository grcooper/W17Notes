\documentclass[12pt]{article}

\setlength\parindent{0pt}
\newcommand{\myt}[1]{\textbf{\underline{#1}}}

\usepackage{mathtools}
\usepackage{amssymb}
\usepackage{graphicx}

\title{\vspace{-15ex}Amath 250 Lecture 2\vspace{-1ex}}
\date{January 4th, 2017}
\author{Graham Cooper}

\begin{document}
	\maketitle
	
	\myt{Seperable Equations}\\
	
	A Differential equation $\frac{dy}{dx} = f(x,y)$ said to be \underline{seperable} if $f(x)$ can be factored as $f(x,y) = g(x)h(y)$\\
	
	In these cases we have $\frac{dy}{dx} = g(x)h(y)$, and dividing by $h(y)$ gives:\\ $$\frac{1}{h(y)}\frac{dy}{dx} = g(x)$$
	Integrating both sides with respect to x gives\\
	$$\int \frac{1}{h(y)} \frac{dy}{dx}dx = \int g(x)dx$$
	
	That is:\\
	
	$$\int \frac{1}{h(y)}dy = \int g(x)dx$$
	
	If we can find antiderivatives and solve for y, we'll have our solution.
		
	Aside (substitution?):\\
	
	$$\int \frac{1}{h(y)}\frac{dy}{dx}$$
	let u = y $du = \frac{dy}{dx}dx$\\
	$$= \int \frac{1}{h(u)}du$$
	$$= \int \frac{h(y)}dy$$
	
	Example: solve $\frac{dy}{dx} = e^{x+y}$\\
	
	This is $\frac{dy}{dx} = d^xe^y$\\
	Trating dy and dx as differentials and seperating the variables we get $\frac{dy}{e^y} = e^xdx$\\
	Summing both sides, we have $\int e^{-y}dy = \int e^ydx$\\
	
	Integrate:\\
	$$-e^{-y} = e^x + c$$
	sovle for y:\\
	$$e^{-y} = -e^x + c_1 (c_1 = -c)$$
	$$-y = ln(c_1 - e^x)$$
	$$y = -ln(c_1 - e^x)$$
	$$= ln[\frac{1}{c_1 - e^x}]$$
	
	Check our answer?:\\
	$$y = ln[\frac{1}{c_1 - e^x}]$$
	$$(c_1 - e^x)[-(\frac{1}{c_1 - e^x})]^2(-e)^x = \frac{e^x}{c_1 - e^x}$$
	$$e^{x+y} = e^xe^y = e^x[\frac{1}{c_1-e^x}] = \frac{e^x}{c_1 - e^x} = \frac{dy}{dx}$$
	
	What if we need the particular solution passing through (0,0)?\\
	From $-e^{-y} = e^x + c$, setting x = y = 0 gives $c = -2$ so $c_1 = 2$, and $y = ln[\frac{1}{2-e^x}]$
	
	You will always have multiple curves, we will want to find the curve that solves for the points given. See image 2.1, the red line is the correct one.\\\\
	
	\myt{One problem to watch for}\\
	
	When seperating variables, we may losee certain "singular" solutions.\\
	Eg. Consider: $\frac{dy}{dx} = -4xy^2$\\
	We may write (if $y != 0$)\\
	$$\int \frac{dy}{y^2} = -\int 4xdx$$
	$$\frac{-1}{y} = -2x^2 + c$$
	$$y = \frac{1}{2x^2-c}$$
	
	This is the general solution, and yet y = 0 is \underline{also} a solution\\
	
	\myt{First-Order Linear Equations}\\
	
	A first-order Differential equation is said to be \underline{linear} if it is of the form $a_1(x)\frac{dy}{dx} + a_0(x)y = f(x)$\\
	
	To motivate our method of solution, consider the special case where $a_0x = a_1'(x)$. 
	$$a_1y' + a_1'y = f(x)$$
	
	We recognize the LHS as $\frac{d}{dx}[a_1(x)y(x)]$\\
	$$\frac{d}{dx}[a_1y] = f(x)$$
	$$a_1y = \int f(x)dx$$
	$$y(x) = \frac{1}{a_1(x)}\int f(x)dx$$
	
	What if $a_0 \ne a_1'$? We can actually \underline{create} this structure! We'll multiply through by another function I(x) (an integrating factor).\\
	\textbf{Step 1} Divide trhough by $a_1(x)$ to get:\\
	$$\frac{dy}{dx} + k(x)y = g(x)$$
	Where $k = \frac{a_0}{a_1}$ and $g = \frac{f}{a_1}$\\
	This is referred to as the \underline{standard form} of a linear first-order differential equation\\
	
	\textbf{Step 2} Multiply through by the (unknown) factor I(x)\\
	$$I(x)\frac{dy}{dx} + I(x)k(x)y = I(x)g(x)$$
	Now we want $I(x)k(x)$ to be equal to $\frac{dI}{dx}$\\
	ie:
	$$ \frac{dI}{dx} = KI$$
	$$\int\frac{dI}{I} = \int K(x)dx$$
	$$ln|I| = \int K(x)dx$$
	$$|I| = e^{\int K(x)dx}$$
	$$I = e^{\int K(x)dx}$$
	(aside) $e^{H(x) + c} = e^ce^{H(x)}$ so we can remove the absolutes?\\
	We now have:\\
	$$\frac{d}{dx}[Iy] = I(x)g(x)$$
	$$y = \frac{1}{I(x)}\int I(x)g(x)dx$$
	where $I = e^{\int k(x)dx}$\\
	
	\myt{Jan 9th:}
	$$a_1(x)\frac{dy}{dx} + a_0(x)y = f(x)$$
	
	\begin{enumerate}
		\item Put the equation in standard form: $\frac{dy}{dx} + k(x)y = g(x)$
		\item Find the integrating factor \\
		$I(x) = e^{\int k(x)dx}$\\
		$= e^{K(x) + C}$\\
		$= e^ce^{K(x)}$\\
		\item Factor this into the equation, and recall that the LHS must be $\frac{d}{dx}(I(x)y(x))$\\
		$I(x)\frac{dy}{dx} + I(x)k(x)y = I(x)g(x)$\\
		$\frac{d}{dx}(Iy) = I(x)g(x)$
	\end{enumerate}

	\myt{examples:}\\
	1.\\
	$\frac{dy}{dx} + xy = x$ (already in standard form)\\
	An integrating factor is $I(x) = e^{\int k(x)dx} = e^{\int xdx} = e^{\frac{1}{2}x^2}$\\
	
	Incorporating this, we have:\\
	$$e^{\frac{1}{2}x^2}\frac{dy}{dx} + xe^{\frac{1}{2}x^2}y = xe^{\frac{1}{2}x^2}$$
	$$\frac{d}{dx}(e^{\frac{1}{2}x^2}y) = xe^{\frac{1}{2}x^2}$$
	
	Make sure we have the correct integrating factor, make sure that the above is the derivative of the line below. CHECK THIS ALWAYS.
	
	$$e^{\frac{1}{2}x^2}y = \int xe^{\frac{1}{2}x^2}dx$$
	$$= e^{\frac{1}{2}x^2} + c$$
	$$=> y = 1 + Ce^{-\frac{1}{2}x^2}$$
	
	2.\\
	
	$$xlnx\frac{dy}{dx} + y - x^3lnx = 0$$
	(assume $x > 1$)\\
	In standard form:\\
	$$\frac{dy}{dx} + \frac{y}{xlnx} = x^2$$
	Int. Factor: $I(x) = e^{\int \frac{1}{xlnx}dx}$\\
	$= |lnx| = lnx $(since x $>$ 1) (see below how we found this)\\
	$$\int \frac{1}{xlnx}dx$$
	$$= \int \frac{1}{u} du$$
	$$= ln|u| + c$$
	$$= ln|lnx| + c$$
	
	\_\_\_\_\_\_\_
	
	$$lnx\frac{dy}{dx} + \frac{y}{x} = x^2lnx$$
	$$\frac{d}{dx}((lnx)y) = x^2lnx$$
	Check?
	$$ylnx = \int x^2lnxdx$$
	$$ylnx = uv - \int vdu$$
	$$= \frac{1}{3}x^3lnx - \int \frac{x^2}{3}dx$$
	$$= \frac{1}{3}x^3lnx = \frac{x^3}{9} + C$$
	$$y = \frac{1}{3}x^3 - \frac{x^3}{9lnx} + \frac{c}{lnx}$$
	
	where above, $u = lnx$ $du = \frac{1}{x}dx$ $dv = x^2dx$ $v = \frac{1}{3}x^3$\\
	
	
\end{document}
