\documentclass[12pt]{article}

\setlength\parindent{0pt}
\newcommand{\myt}[1]{\textbf{\underline{#1}}}

\usepackage{mathtools}
\usepackage{amssymb}
\usepackage{graphicx}

\title{\vspace{-15ex}CS370 Lecture 1\vspace{-1ex}}
\date{January 4th, 2017}
\author{Graham Cooper}

\begin{document}
	\maketitle
	\myt{Five Topics in the Course (Jan 4th)}
	\begin{itemize}
		\item Floating point numbers and Arithmetic
		\item Iterpolation, Splines, Parametric Curves
		\item Initial Value Problems - solve differencial equations
		\item Discrete Fourier Analysis
		\item Numerical Linear Algebra - solve equations - google pagerank
	\end{itemize}

	\section*{Topic 1: Floating Point Arithmetic}
	Examples where problems come whehn using approximation\\
	
	eg1. $e^{-5.5} = $\\
	
	$$e^x = 1 + x + \frac{x^2}{2!} + \frac{x^3}{3!} + ... + \frac{x^n}{n!} + ...$$
	$$e^{-x} = \frac{1}{x}$$
	$$e^{-5.5} = \frac{1}{e^{5.5}} = \frac{1}{1 + 5.5 + \frac{5.5^2}{2} + ...}$$
	
	Now do arithmetic keeping only 5 digits. In both cases infinite sums remain unchanged after 25 terms. There is no sense in going any further - we end up just truncating all of the terms after this as they are smaller than the 5th digit.\\
	
	Method 1 gives $e^{-5.5} = 0.0026363$\\
	Method 2 gives $e^{-5.5} = 0.0040868$\\\\
	
	eg2. $ax^2 + bx + c = 0$\\
	1
	$$x_1 = \frac{-b +\sqrt{b^2-4ac}}{2a}$$
	$$x_2 = \frac{-b -\sqrt{b^2-4ac}}{2a}$$
	2
	$$x_1 = \frac{2a}{-b +\sqrt{b^2-4ac}}$$
	$$x_2 = \frac{2a}{-b -\sqrt{b^2-4ac}}$$
	
	$$x^2 + 621x + 1 = 0$$
	use 4 digit arithmetic\\
	
	1: $x_1 = -0.200$ $x_2 = 62.1$\\
	2: $x_1 = -0.0161$ $x_2 = -62.1$\\
	
	\subsection*{Floating point number systems (Jan 6th)}
	
	$$F(\beta , t, L, u)$$
	Floating point number is either\\
	$$0 +/- 0.x_1x_2...x_t$$
	Where $x_n$ is a digit\\
	\begin{itemize}
		\item $x_1 \ne 0$
		\item $0 \le x_i \textless\beta$
		\item $L \le d \le u$
	\end{itemize}
	
	eg $F(10, 6, L, u)$ for $\pi = 0.314519 x 10^1$\\
	
	Single Precision: [+/-][8 bits][23 digits] F(2, 24, -126, 127)\\
	Double Precision: [+/-][11 bits][52 digits] F(2, 53, -1022, 1023)\\
	
	Given any real number:
	$$+/- 0.x_1x_2...x_tx_{t+1}...x\beta ^d$$
	you can truncate (rounding possible):
	$$fl(x) = +/- 0.x_1x_2...x_t$$
	
	\myt{Question:} How close is fl(x) to x? Relatively!\\
	
	$$\delta = \frac{fl(x) - x}{x}$$
	Claim: $|\delta| \textless \beta^{1-t}$ when truncating and $|\delta| \textless \frac{1}{2}\beta^{1-t}$ when rounding (Let this definition be $\epsilon$)
	
	Trucating:
	$$|\delta| = \frac{0.00...0x_{t+1}...x\beta^d}{0.x_1x_2...x\beta^d}$$
	$$|\delta| = \frac{0.x_{t+1}x_{t+2}...x\beta^{-t}}{x_1.x_2x_3...x\beta^{-1}}$$
	$$|\delta| \textless \beta^{1-t}$$
	
	In general $|\delta| \textless \epsilon$\\
	
	$$fl(x) = x(1 + \delta)$$
	$$|\delta| \textless \epsilon$$
	
	What about arithmetic an errors?\\
	
	x,y real\\
	$x \oplus y = $
	
	eg.
	$x = 0.1111...$ F(2,4,L,u)\\
	$y = 0.1110...$\\
	
	$fl(x) = 0.1111$\\
	$fl(y) = 0.1110$\\
	when adding them: $1.1101$ then truncate to $0.1110 \times 2^1$\\
	
	x,y real\\
	$x \oplus y = fl(fl(x) + fl(y))$
	
	$$|\frac{x \oplus y - (x + y)}{x + y}|$$
	$$ = |\frac{fl(fl(x) + fl(y)) - (x + y)}{x + y}|$$
	since $|\delta_1| < \epsilon$ and $|\delta_2| < \epsilon$ and $|\delta_3| < \epsilon$\\
	$$=|\frac{(x(1 + \delta_1) + y(1+\delta_2))(1+\delta_3) - (x+y)}{x + y}|$$
	$$|\frac{x + y + x\delta_1 + y\delta_2 + x\delta_3 + y\delta_3 + x\delta_1\delta_3 + y\delta_2\delta_3 - x - y}{x+y}|$$
	$$|\frac{x\delta_1 + y\delta_2 + x\delta_3 + y\delta_3 + x\delta_1\delta_3 + y\delta_2\delta_3}{x+y}|$$
	$$\le \frac{|x||\delta_1| + |y||\delta_2| + |x||\delta_3| + |x||\delta_1||\delta_3| + |y||\delta_2\delta_3|}{|x+y|}$$
	Since $|a+b| \le |a|+|b|$ and $|a \times b| = |a||b|$
	$$< \frac{(|x| + |y|)}{|x+y|}(2\epsilon + \epsilon^2)$$
	$$|\frac{x\oplus y - (x+y)}{x+y}|$$
	$$ < \frac{|x| + |y|}{|x+y|}(2\epsilon + \epsilon^2)$$
	
	If x and y have the same sign then $|x + y| = |x| + |y|$ and so addition has relative error bounded by $2\epsilon + \epsilon^2$\\
	
	But if x and y are of opposite sign and perhaps of nearly the same size:
	$$\frac{|x| + |y|}{|x + y|}$$
	CATASTROPHIC CANCELLATION - Subtraction is deadly.
	
	\myt{Jan 9th:}
	
	Recall: $F(\beta, t, L, u)$\\
	For any non-zero real number\\
	$x = 0.x_1x_2...x\beta^d$\\
	$fl(x) = 0.x_1x_2...x_t$ then the exponent $x\beta^d$\\
	
	We should the magnitude of the error:\\
	$\delta = \frac{fl(x) - x}{x}$\\
	showed: $|\delta| < \epsilon = \beta^{1-t}$ or $\epsilon = \frac{1}{2}\beta^{1-t}$\\
	$\epsilon$ is called machine epsilon.\\
	
	$fl(x) = x \dot (1 + \delta)$ where $|\delta| < \epsilon$\\
	
	\subsection*{Operations}
	If x and y are two real numbers\\
	$x \oplus y = fl(fl(x) + fl(x))$\\
	
	$$|\frac{(x \oplus y) - (x + y)}{x+y} < (2\epsilon + \epsilon^2)(\frac{|x| + |y|}{|x+y|})$$
	
	Notice: if x and y are of the same sign then relative error of addition $< 2\epsilon + \epsilon^2$\\
	
	If x and y are of opposite sign and $x + y \approx 0$ then the upper bound can be significant. This is called catastrophic cancellation\\
	
	$$x = 0.x_1x_2...x_t...x\beta^d$$
	$$y = -0.x_1x_2...\hat{x}_t...x\beta^d$$
	Subtracted $ = 0.0.....0?$ with exponent $x\beta^d$\\
	
	$$x =  0.1239...1231$$
	$$y = -0.1229...1221$$
	Subtracted $= 0.0010$\\
	
	$$e^{-5.5} = 1 - 5.5 + \frac{5.5^2}{2} - \frac{5.5^3}{6} + ... = x - \hat{x}$$
	$$e^{-5.5} = \frac{1}{e^{5.5}} = \frac{1}{1 + 5.5 + \frac{5.5^2}{2} + ...}$$	
	
	\subsection*{Numerical Algorithm Issues}
	
	\myt{Problem}: given $\alpha$ Solve numerically:\\
	
	$$I_n = \int_{0}^{1}\frac{x^n}{x+\alpha}dx$$
	n = 0,1,2...\\

	\myt{Method}\\
	$$I_0 = \int_{0}^{1}\frac{1}{x+\alpha}dx = ln(x+\alpha)|_0^1$$
	$$ = ln(1+\alpha) - ln(\alpha) = ln(\frac{1+\alpha}{\alpha})$$
	$$I_n = \frac{1}{n} - \alpha I_{n-1}$$
	For $n \ge 1$:\\
	$$I_n = \int_{0}^{1}\frac{x^{n-1}(x + \alpha - \alpha)}{x+\alpha}dx$$
	$$= \int_{0}*{1}x^{n-1}dx - \alpha\int_{0}^{1}\frac{x^{n-1}}{x+\alpha}dx$$
	
	$\alpha = 0.5$\\
	$$I_0 = ln(3) = 1.09861228866..$$
	$$I_0^app = 1.098612$$
	$$I_100^app = 0.00664$$
	
	$\alpha = 2.0$\\
	$$I_0 = ln(1.5) = 0.405465108108..$$
	$$I_0^app = 0.4054654$$
	$$I_100^app = 2.1 \times 10^{22}$$
	
	He used his own application and went through the solutions to the above. They are incorrect :S\\
	
	Math: $I_0, I_n = \frac{1}{n} - \alpha I_{n-1}$\\
	CS: $I_0^{app}, I_n^{app} = \frac{1}{n} - \alpha I_{n-1}^{app}$\\
	
	$e_0 = I_0 - I_0^{app} = (\frac{1}{n} - \alpha I_{n-1}) - (\frac{1}{n} - \alpha I_{n-1}^{app})$\\
	$= -\alpha I_{n} + \alpha I_{n-1}^{app} = -\alpha e_{n-1}$\\
	
	$$e_n = \alpha e_{n-1}$$
	$$= \alpha^2 e_{n-2}$$
	$$= - \alpha^3 e_{n-3}$$
	$$...$$
	$$= (-\alpha)^ne_0$$
	
	IF $|\alpha| < 1$ then $|e_n| \rightarrow 0$ as $n \rightarrow \infty$ stable\\
	$|\alpha| > 1$ then $|e_n| \rightarrow \infty$ is $n \rightarrow \infty$ unstable!!!\\
	
	When $|\alpha| > 1$ then we can work backwards!\\
	$$I_{200}, I_{199}, ..., I_{101}I_{100}$$
	$$I_{n-1} = \frac{1}{\alpha n} - \frac{1}{\alpha}I_n$$
	$$e_{n-1} = \frac{1}{2}e_n$$
	$$|e_{199}| = \frac{1}{|\alpha|}|e_{200}|$$
	$$|e_{100}| = \frac{1}{|\alpha|^{100}}|e_{200}|$$
	
	
\end{document}
