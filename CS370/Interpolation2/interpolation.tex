\documentclass[12pt]{article}

\setlength\parindent{0pt}
\newcommand{\myt}[1]{\textbf{\underline{#1}}}

\usepackage{mathtools}
\usepackage{amssymb}
\usepackage{graphicx}
\usepackage{amsmath}

\title{\vspace{-15ex}CS370: Interpolation \vspace{-1ex}}
\date{January 11th, 2017}
\author{Graham Cooper}

\begin{document}
	\maketitle
	
	See figure 2.1\\
	
	$$y = p(x)$$
	We want to find a function p, such that the curve is 'nice' (where nice is \underline{piecewise polynomial} or \underline{polynomial})\\
	
	Given:\\
	$(x_1, y_1), (x_2,y_2)...,(x_n, y_n)$ n points $ x_1 < x_2 < ... < x_n$\\
	
	Find a polynomial P(x) of degree $< n$\\
	In general:\\
	$$p(x) = c_1 + c_2x + c_3x^2 + ... + c_nx^{n-1}$$
	$$p(x_1) = y_1$$
	$$p(x_2) = y_2$$
	$$...$$
	$$p(x_n) = y_n$$
	
	n unknowns, n equations (linear)\\
	
	\myt{Example:}\\
	$(-1, 1), (1,1), (2,5), (4,1)$\\
	See figure 2.2\\
	
	$$p(x) = c_1 + c_2x + c_3x^2 + c_4x^3$$
	$$p(-1) = c_1 - c_2 + c_3 - c_4 = 1$$
	$$p(1) = c_1 + c_2 + c_3 + c_4 = 1$$
	$$p(2) = c_1 + 2c_2 + 4c_3 + 8c_4 = 5$$
	$$p(4) = c_1 + 4c_2 + 16c_3 + 64c_4 = 1$$
	
	$
	\begin{Bmatrix}
		1 & -1 & 1 & -1 & | & 1 \\
		1 & 1 & 1 & 1 & | & 1 \\
		1 & 2 & 4 & 8 & | & 5 \\
		1 & 4 & 16 & 64 & | & 1
	\end{Bmatrix}
	$//
	Solve the matrix!!\\
	
	Now we are just writing out the solution...\\
	$$p(x) = c_1 + c_2x + c_3x^2 + c_4x^3$$
	$$= 1 + b_2(x-1)+b_3(x-1)^2+b_4(x-1)^3$$
	$$= L_1(x) + L_2(x) + 5L_3(x) + L_4(x)$$
	
	$$L_1(x) = \frac{(x-1)(x-2)(x-4)}{-30}$$
	$$L_2(x) = \frac{(x+1)(x-2)(x-4)}{6}$$
	$$L_3(x) = \frac{(x+1)(x-1)(x-4)}{-6}$$
	$$L_4(x) = \frac{(x+1)(x-1)(x-2)}{30}$$
	
	I think we are writing it out this way so that we can easily plug in the vlues and get the correct points??
	
	Question:\\
	\begin{enumerate}
		\item Does an interpolating polynomial always exist?
		\item If (1) is true then is the answer always unique?
	\end{enumerate}

	$$p(x) = c_1 + c_2x + ... c_nx^{n-1}$$
	$$p(x_1) = c_1 + c_2x_1 + ... c_nx_1^{n-1}$$
	$$p(x_2) = c_1 + c_2x_2 + ... c_nx_2^{n-1}$$
	$$...$$
	$$p(x_n) = c_1 + c_2x_n + ... c_nx_n^{n-1}$$
	
	$
	\begin{Bmatrix}
		1 & x_1 & x_1^2 & ... & x_1^{n-1} \\
		1 & x_2 & x_2^2 & ... & x_2^{n-1} \\
		... & ... & ... & ... & ... \\
		1 & x_n & x_n^2 & ... & x_n^{n-1} \\
	\end{Bmatrix}
	\begin{Bmatrix}
		c_1 \\ c_2 \\ ... \\ c_n
	\end{Bmatrix}
	 = 
	 \begin{Bmatrix}
		 y_1 \\ y_2 \\ ... \\ y_n
	 \end{Bmatrix}
	$
	
	The first matrix is the vandermonde (V) matrix.\\
	V is invertable, $V \times \overrightarrow{c} = \overrightarrow{y}$\\
	det $V \ne 0$ and det $V = \pi(x_i - x_j) \ne 0$ for $i < j$\\
	
	Remember what a determinate is, remember what invertible is, but we will never be asked to do it.\\
	
	$p(x)$
	$$p(x) = q_1(x)(x-x_1) + y_1$$
	$$p(x) = q_2(x)(x-x_2) + y_2$$
	$$...$$
	$$p(x) = q_n(x)(x-x_n) + y_n$$
	
	\subsection*{Lagrange Polynomial}
	
	$$(x_1, y_1), (x_2,y_2)...(x_n,y_n)$$
	$$p(x) = y_1L_1(x) + y_2L_2(x) + ... + y_nL_n(x)$$
	
	$L_i(x_i) = 1, L_i(x_j) = 0$ for $i \ne j$ and $deg(L_i) = n-1$\\
	Lets construct $L_1$ using the above\\
	$$L_1(x) = \frac{(x-x_2)(x-x_3)...(x-x_n)}{(x_1-x_2)(x_1-x_3)...(x_1-x_n)}$$
	
	$$L_i(x) = \frac{(x-x_1)...(x-x_{i-1})(x-x_{i+1})...(x-x_n)}{(x_i-x_1)...(x_i-x_{i-1})...(x_i-x_n)}$$
	$L_i(x_i) = 1$ and $L_j(x_j) = 0$ where $j \ne i$\\
	
	For A1 Q3 (January 13th) - figuring out the solution to the recurrence - and using the answer to help\\
	$$?? \boxed{I_n} \leftarrow I_{n-1} \leftarrow I_{n-2} \leftarrow ... \leftarrow I_0$$
	$$\checkmark \boxed{\hat{I_n}} \leftarrow \hat{I_{n-1}} \leftarrow ... \leftarrow \hat{I_1} \leftarrow \hat{I_0}$$
	$$e_n \leftarrow e_{n-1} \leftarrow ... \leftarrow e_1 \leftarrow e_0$$
	$$e_n = (-\alpha)^ne_0$$
	$$I_n ?= formula(I_0) = $$
	
	Using p?\\
	$$?? \boxed{p_n} \leftarrow p_{n-1} p_{n-2}, p_{n-2}p_{n-3}, ..., p_1, p_0$$
	$p_n = as^n + bt^n$ and $a,b$ depend on $p_0, p_1$\\
	$$\checkmark \boxed{\hat{p_n}} \leftarrow \hat{p_{n-1}} \hat{p_{n-2}} ..., \hat{p_1}\hat{p_0}$$
	This line but with hats (I got lazy)  $p_n = as^n + bt^n$ and $a,b$ depend on $p_0, p_1$\\
	solve for $e_n$\\
	
	Recall from Jan 11th: (regoing over the start of this page)\\
	
	\subsection*{Lagrange Form (again)}
	For $x_1, x_2, ... x_n$ distinct, construct $L_1(x), L_2(x)...L_n(x)$\\
	Satisfying:\\
	\begin{enumerate}
		\item $L_i(x)$ has degree n-1
		\item $L_i(x_i) = 1$
		\item $L_i(x_j) = 0$ if $i \ne j$
	\end{enumerate}

	How do we construct this:\\
	$$L_1(x) = \frac{(x-x_2)(x-x_3)...(x-x_n)}{(x_1-x_2)(x_1-x_3)...(x_1-x_n)}$$
	We divide like this in order to get an equation that satisfies that if we plug in $x_1$ we will end up getting 1 as required, otherwise we will be getting a 0. This is actually pretty cool. Neat!\\
	
	$$L_i(x) = \frac{(x-x_1)...(x-x_{i-i})(x-x_{i+1})...(x-x_n)}{(x_i-x_1)...(x_i-x_{i-1})(x_i-x_{i+1})...(x_i-x_n)}$$
	
	$$p(x) = y_1L_1(x) + y_2L_2(x) + ... + y_nL_n(x)$$
	$$p(x_1) = y_11 + y_20 + ... + y_n0 = y_1$$
	$$...$$
	$$p(x_n) = y_10 + y_20 + ... + y_n1 = y_n$$
	
	A question that he often has asked on midterms ( and is almost 100\% going to add it to ours):\\
	Given: $x_1,x_2,x_3x_4$ as $-1,1,2,117,412$\\
	Form $p(x) = L_1(x) + L_2(x) + L_3(x) + L_4(x)$\\
	Write $p(x) = c_1 + c_2x + c_3x^2 + c_4x^3$\\
	
	Draw the graph!\\
	Solve for the 4 numbers, and find what is y at each of the 4 points?\\
	Then we find out that f(x) = 1 for each\\
	Therefore the solution is p(x) = 1\\
	
	\subsection*{Cubic Hermite Interpolation}
	Another type of interpolation\\
	
	Given: $(x_L, y_L)$ more on the left side and $(x_R, y_R)$ on the right side, $S_L$ slope of the left side, and $S_R$ the slope of the right side\\
	
	$p(x)$ has degree at most 3 since we have 4 uknowns\\
	$p(x_L) = y_L$, $p(x_R) = y_R$, $p'(x_L) S_L$, $p'(x_R) = S_R$\\
	
	$p(x) = c_1 + c_2(x-x_L) + c_3(x-x_L)^2 + c_4(x-x_L)^3$
	$p'(x) = c_2 + 2c_3(x-x_L) + 3c_4(x-x_L)^2$
	$p(x_L) = y_L \implies c_1$
	$p'(x_L) = S_L \implies c_2$
	$p(x_R) = y_R \implies c_1 + c_2\Delta x + c_3\Delta x^2 + c_4 \Delta x^3 = y_R$
	$p'(x_R) = S_R \implies c_2 + 2c_3\Delta x + 3c_4\Delta x^2 = S_R$
	where $\Delta x = x_R - x_L$
	
	$
	\begin{Bmatrix}
	 1 & 0 & 0 & 0 & | & Y_L \\
	 0 & 1 & 0 & 0 & | & S_L \\
	 1 & \Delta x & \Delta x^2 & \Delta x^3 & | & Y_R \\
	 0 & 1 & 2\Delta x & 3\Delta x^2 & | & S_L \\
	\end{Bmatrix}
	$\\
	becomes\\ 	
	$
	\begin{Bmatrix}
	1 & 0 & 0 & 0 & | & Y_L \\
	0 & 1 & 0 & 0 & | & S_L \\
	0 & 0 & 1 & 0 & | & \frac{3Y_R' - 2S_L - S_R}{\Delta x} \\
	0 & 0 & 0 & 1 & | & \frac{S_R + S_L - 2y_L'}{\Delta x^2} \\
	\end{Bmatrix}
	$
	
	$$c_1 = y_L$$
	$$c_2 = S_L$$
	$$c_3 = \frac{3Y_R' - 2S_L - S_R}{\Delta x}$$
	$$c_4 = \frac{S_R + S_L - 2y_L'}{\Delta x^2}$$
	
	Sub into p(x)\\
	$$p(x) = 3 - (x-1) + 3(x-1)^2 - (x-1)^3$$
	
	From Jan 16th:\\
	See image Interp1.1: He is showing that the polynomial (red line) could be bad, we want the green line instead.\\
	
	\subsection*{Cubic Spline}
	Given: $(x_1,y_1),...,(x_N,y_N)$ N points\\
	$x_1 < x_2 < ... < x_{N-1} < x_{N}$\\
	
	A \underline{cubic spline} is a function S(x) defined on the interval $[x_1, x_N]$ which satisfies the following:\\
	(see interp1.2 figure)
	\begin{enumerate}
		\item In each interval $[x_i,x_{i+1}]$ S(x) is a cubic polynomial. $S_i(x) = a_i + b_i(x-x_i) + c_i(x-x_i)^2 + d_i(x-x_i)^3$
		\item S(x) interpolates the N points: $S(x_i) = y_i$
		\item S'(x) is continuous
		\item S''(x) is continuous
		\item 2 other things??
	\end{enumerate}
	\myt{Is this well defined?}\\
	How man unknowns? 4 per interval, N-1 intervals $\rightarrow 4N-4$ unknowns\\
	How many conditions (equations)?\\
	Condition(2) $\rightarrow$ 2 equations per interval $\rightarrow 2N-2$\\
	$S_i(x_i) = y_i$ and $S_i(x_{i+1}) = y_{i+1}$\\
	Condition(3) - 1 equation per interior point $\rightarrow N-2$\\
	Condition(4) - 1 equation per interiour point $\rightarrow N-2$\\
	In total we get $4N-6$ equations\\
	
	\myt{Boundary Conditions}
	\begin{enumerate}
		\item Natural cubic spline $S''(x_1) = 0$, $S''(x_N) = 0$
		\item Clamped cubic spline $S'(x_1) = s_1$ and $S'(x_N) = s_N$ $s_1, s_N$ are known
		\item Periodic Cubic spline $S'(x_N) = S'(x_1)$ and $S''(x_N) = S''(x_1)$
		\item Not-a-knot condition (Matlab default) $S'''(x)$ is continuous at $x_2$ and $x_{N-1}$
	\end{enumerate}

	\subsubsection*{How do we compute a cubic spline?}
	\myt{Method 1}:\\
	Have $4N-4$ uknowns and $4N-4$ linear equations $\rightarrow$ sove via Gaussian elimination\\
	This is a cost of: $O((4N-4)^3) = O(N^3)$\\
	
	\myt{Method 2}:\\
	Think of the deriviatives $S_1, S_2, ... S_N$ as the unknowns. We will set up linear equations for these derivatives\\
	Then:\\
	\begin{enumerate}
		\item This will give us $S_1(x),S_2(x), ..., S_{N-1}(x)$
		\item We will solve linear system in O(N) operations
	\end{enumerate}


	
	
\end{document}
