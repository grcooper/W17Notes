\documentclass[12pt]{article}

\setlength\parindent{0pt}
\newcommand{\myt}[1]{\textbf{\underline{#1}}}

\usepackage{mathtools}
\usepackage{amssymb}
\usepackage{graphicx}

\title{\vspace{-15ex}CS343 Lecture 1\vspace{-1ex}}
\date{January 4th, 2017}
\author{Graham Cooper}

\begin{document}
	\maketitle
	
	\myt{Advanced Control Flow}
	
	Things that are required in the course - docked marks.\\
	\begin{itemize}
		\item within a routine, basic and advanced control structurs allow virtually any control flow
		\item multi-exit-loop loop, why would you want to put a break in a loop? We end up with code that is duplicated - change one and forget about the other. For loop can be used to add an index easily for( ;; ) into for( int x = 0;; x++)
		\item TFW flagism is a thing - don't use flags :O
		\item eliminiate variables necessary with while
		\item use labbeled exits, which are pretty cool
		\item write defensive code - as you add code in the future, how can you reduce the amount of pain then, by coding smart right now?
		\item only use goto to perform static multi-level exit, eg. simulated labelled break and continue for the eye candy. the break and goto are dangerous except:
		\begin{itemize}
			\item cannot loop (only forward branch) therefore only loop constructs branch back
			\item cannot branch into a control structure
		\end{itemize}
	\end{itemize}

	\myt{Dynamic Memory Allocation}
	
	\begin{itemize}
		\item Stack allocation eliminates explicit storage management and often more efficent than heap allocation - "Use the stakc luke skywalker"
		\item rule of using the heap: 
		\begin{itemize}
			\item If the lifetime of the storage exceeds the block that it is declared in, then it must be declared on the heap. Otherwise USE THE STACK
			\item you might also want to use the heap when the amount of data you read is unknown
			\item when an array of objects must be initialized via the object's constructor
			\item when large local variables are allocated on a small stack.
		\end{itemize}
	\end{itemize}
	
	
\end{document}
