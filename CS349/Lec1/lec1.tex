\documentclass[12pt]{article}

\setlength\parindent{0pt}
\newcommand{\myt}[1]{\textbf{\underline{#1}}}

\usepackage{mathtools}
\usepackage{amssymb}
\usepackage{graphicx}

\title{\vspace{-15ex}CS 349 Lecture 1\vspace{-1ex}}
\date{January 4th, 2017}
\author{Graham Cooper}

\begin{document}
	\maketitle
	
	Red room batch computing came in around 1965, end of batch interfaces. \\
	
	\myt{What is a User Interface}
	\begin{itemize}
		\item Maybe, "The human's view of the computer"
		\item Maybe, "The place where humans and computers meet"
		\item it is a person requesting intent to an artifact, and that artifact returning an option???
	\end{itemize}

	There is a language that needs to be stated between the software and the user. There is a mental model on the user and there is a model within the system. The user expresses intent and the controller decodes or translates it, the model gets the data and changes it if needed, then we notify the view that the value has changed. Then the user sees the change in the view.
	
	\myt{interface vs interaction}
	\begin{itemize}
		\item what is the difference?
		\item interface refers to the external presentation to the user. The play button or like button is an interface.
		\item interaction --- pressing the above buttons to play the music, or skip or like is the actual interaction
	\end{itemize}

	Why is interaction design so hard?\\
	Challenging because of variability in users and tasks, ages are an issue. There is no right way to design an interface - they can always be improved - pushing tech forward requires us to rethink interaction.\\\\
	
	\myt{why study interaction?}
	\begin{itemize}
		\item emporing people
		\item gives people the ability to do things they couldnt do before
		\item interaction is the key to enabling new technologies
		\item a well designed tool can literally change the world
	\end{itemize}
	
	
\end{document}
